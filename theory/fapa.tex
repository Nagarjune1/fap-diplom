\documentclass[a4paper,10pt]{article}
\usepackage[utf8x]{inputenc}
\usepackage[czech]{babel}

\usepackage{amsthm}
\usepackage{amsmath}
\usepackage{amsfonts}
\usepackage{mathtools}

%% číslovaný cases
\usepackage[subnum]{cases}

%% colspan a rowspan v tabulkách
\usepackage{multirow}

%%  http://tex.stackexchange.com/a/361157
\usepackage{showlabels}

%% subfigures, https://en.wikibooks.org/wiki/LaTeX/Floats,_Figures_and_Captions#Subfloats
\usepackage{subcaption}

%% číslování řádků, https://texblog.org/2012/02/08/adding-line-numbers-to-documents/
\usepackage{lineno}

%Definice, věta, důkaz!
\newtheorem{definition}{Definice}
\newtheorem{theorem}{Věta}
\newtheorem*{notation*}{Značení}
\newtheorem{note}{Poznámka}
\newtheorem{example}{Příklad}


\numberwithin{definition}{section}
\numberwithin{theorem}{section}
%\numberwithin{notation}{section}
\numberwithin{note}{section}
\numberwithin{example}{section}


%Ostatní makra
\newcommand{\TODO}[1]{ \textit{\small(zde bude doplněno: #1)} }	
\newcommand{\RLY}{?? }

\newcommand{\lattL}{\mathbb{L}}
\newcommand{\fsubsets}[1]{\mathcal{F}(#1)}

\newcommand{\term}[1]{\textit{#1}}
\newcommand{\str}[1]{{\ttfamily#1}}
\newcommand{\regex}[1]{{\ttfamily#1}}

\newcommand{\impl}[2]{{\ttfamily#1/}{\ttfamily test/data/}{\ttfamily#2}}

% Speciální nastavení

% vycentrované floating objekty
% http://tex.stackexchange.com/questions/2651/should-i-use-center-or-centering-for-figures-and-tables
\makeatletter
\g@addto@macro\@floatboxreset\centering
\makeatother

\linenumbers
\modulolinenumbers[10]


%opening
\title{Aplikace fuzzy a pravděpodobnostních automatů}
\author{Martin Jašek}
\date{12. září 2016 --- ??}

\begin{document}

\maketitle
\tableofcontents
\newpage

%%%%%%%%%%%%%%%%%%%%%%%%%%%%%%%%%%%%%%%%%%%%%%%%%%%%%%%%%%%%%%%%%%%%%%%%%%%%%%%
\section{Definice a značení}
Tato kapitola zatím poslouží jako \uv{skladiště} pro definice a zavedení značení pro ostatní kapitoly.
%V této kapitole budou rozebrány různé definice fuzzy automatů, jejich vztahy (který je speciálním případem kterého apodob.), vlastnosti... Aktuálně však složí jen jako \uv{skladiště} definic, aby bylo na co se odkazovat v ostatních kapitolách.

\subsubsection*{Abecedy, řetězce, jazyky}
Abecedy budou značeny standardně, tedy velkými řeckými písmeny (typicky $\Sigma$). Řetězce pak malými písmeny ($\omega, \alpha, \dots$). Jazyky velkými kaligrafickými písmeny. Jazyk přijímaný automatem $A$ bude značen $\mathcal{L}(A)$.

\subsubsection*{Fuzzy teorie}
Fuzzy množiny a relace budou po vzoru \cite{AstGonMenGar-FuzzAutEpsMovCmpFuzzMeasBtwStrs} nejčastěji malými řeckými písmeny. Množinu všech fuzzy podmnožin množiny $S$ budeme značit $\fsubsets{S}$.

\subsubsection*{Deterministický bivalentní automat}
\TODO{zdroj: Eilenberg S.: Automata, Languages and Machines, Vol. A, Academic Press, New York, 1974. Pokud ji někde seženu (odkazuje se na ni Bel v \cite{Bel-FuzRelSyss}}
\begin{definition}
 Konečný deterministický (bivalentní) automat je pětice $A = ( Q, \Sigma, \delta, q_0, F )$, kde $Q$ je konečná množina stavů, $\Sigma$ je vstupní abeceda, $\delta: Q \times \Sigma \rightarrow Q$ je přechodová funkce, $q_0 \in Q$ je počátační stav a $F \subseteq Q$ je množina koncových stavů.
\end{definition}


\subsubsection*{Nedeterministický bivalentní automat}
\TODO{Značení převzato z FJAA, dohledat zdroj}
\begin{definition}
 Konečný nedeterministický (bivalentní) automat je pětice $A = ( Q, \Sigma, \delta, I, F )$, kde $Q$ je konečná množina stavů, $\Sigma$ je vstupní abeceda, $\delta: Q \times \Sigma \rightarrow 2^Q$ je přechodová funkce, $I \subseteq Q$ je množina počátačních stavů a $F \subseteq Q$ je množina koncových stavů.
\end{definition}


\subsubsection*{Základní definice nedeterministického fuzzy automatu}
Značení je převzato z \cite{AstGonMenGar-FuzzAutEpsMovCmpFuzzMeasBtwStrs} a lehce upraveno.

\begin{definition}[Nedeterministický fuzzy automat]\label{def-ZaklDefNedFuzzAut}
 Nedeterministický fuzzy automat $A$ je pětice $(Q, \Sigma, \mu, \sigma, \eta)$, kde $Q$ je konečná množina stavů, $\Sigma$ je abeceda, $\mu$ je fuzzy přechodová funkce (fuzzy relace $Q \times \Sigma \times Q \rightarrow [0, 1]$) a $\sigma$ a $\eta$ jsou po řadě fuzzy množiny nad $Q$ počátačních, resp. koncových stavů.
\end{definition}

\begin{definition}[Fuzzy stav]\label{def-FuzzStav}
 Mějme nedeterministický fuzzy automat $A$. Pak jako fuzzy stav označujeme fuzzy podmnožinu jeho stavů, tj. $V \in \fsubsets{Q}$.
\end{definition}

\begin{definition}[Aplikace fuzzy relace na fuzzy stav]
 Mějme nedeterministický fuzzy automat $A$ a fuzzy symbol $V$. Pak aplikací binární fuzzy relace $R: Q \times Q \rightarrow [0, 1]$ na fuzzy stav $V$ obdržíme fuzzy symbol $V \circ R$ splňující pro každé $p \in Q$: $(V \circ R )(p) = \max_{q \in Q} (V(q) \otimes R(q, p))$.
\end{definition}

\begin{definition}[Přechodová funkce fuzzy stavů]\label{def-PreFunFuzzStav}
 Mějme nedeterministický fuzzy automat $A$. Pak přechodová funkce fuzzy stavů je fuzzy relace $\widehat{\mu}: \fsubsets{F} \times \Sigma \rightarrow \fsubsets{F}$ taková, že pro každý fuzzy stav $V \in \fsubsets{Q}$ a symbol $x \in \Sigma$ je $\widehat{\mu}(V, x) = V \circ \mu[x]$.
\end{definition}

\begin{note}
 Označení $\mu[x]$ je fuzzy relace, pro kterou platí: $\mu[x](p, q) = \mu(p, x, q)$ pro všechna $x \in \Sigma$ a $p, q \in Q$.
\end{note}

\begin{definition}[Rozšířená přechodová funkce]\label{def-PreFunFuzzStav}
 Mějme nedeterministický fuzzy automat $A$. Pak rozšířená přechodová funkce (fuzzy stavů) je fuzzy relace $\mu^*: \fsubsets{F} \times \Sigma^* \rightarrow \fsubsets{F}$ (\TODO{co je to $F$? Nemá to být $Q$?!}) daná následujícím předpisem:
 \begin{enumerate}
  \item $\mu^*(V, \epsilon) = V$ pro všechna $V \in \fsubsets{Q}$
  \item $\mu^*(V, \alpha x) = \widehat{\mu}(\mu^*(V, \alpha), x)$ pro všechna $V \in \fsubsets{Q}, \alpha \in \Sigma^*, x \in \Sigma$
 \end{enumerate}
\end{definition}

\begin{definition}[Řetězec přijímaný automatem]\label{def-RetPriAut}
 Mějme nedeterministický fuzzy automat $A$. Pak řětězec $\alpha \in \Sigma^*$ je automatem $A$ přijat ve stupni
 $$
  A(\alpha) = \max_{q \in Q} (\mu^*(\sigma, \alpha)(q) \otimes \eta(q))
 $$
 \TODO{ověřit, dohledat, ozdrojovat}
\end{definition}

\begin{definition}[Jazyk rozpoznávaný automatem]\label{def-JazRozpAut}
 Mějme nedeterministický fuzzy automat $A$. Pak fuzzy množinu $\mathcal{L}(A)(\alpha) = A(\alpha)$ nad univerzem $\Sigma^*$ nazýváme fuzzy jazyk rozpoznávaný automatem $A$.

 \TODO{ověřit, dohledat, ozdrojovat}
\end{definition}

\subsubsection*{Nedeterministický fuzzy automat s $\epsilon$ přechody}

\begin{definition}[Nedeterministický fuzzy automat s $\epsilon$ přechody]\label{def-NedFuzzAutEpsPre}
\TODO{dohledat přesně, zkontrolovat a ozdrojovat}
 Nedeterministický fuzzy automat $A$ je pětice $(Q, \Sigma, \mu, \sigma, \eta)$, kde $Q$ je konečná množina stavů, $\Sigma$ je abeceda, $\mu$ je fuzzy přechodová funkce (fuzzy relace $Q \times (\Sigma \cup \{ \epsilon \}) \times Q \rightarrow [0, 1]$) a $\sigma$ a $\eta$ jsou po řadě fuzzy množiny nad $Q$ počátačních, resp. koncových stavů.
\end{definition}

\TODO{Tady by asi bylo vhodné rozebrat $\epsilon$-uzávěry}

\subsubsection*{Konstrukce fuzzy automatu z konečného automatu}
V praxi se často setkáme z problémem, kdy máme k dispozici konečný bivalentní automat avšak my potřebujeme pro naši práci fuzzy automat. Je tedy třeba zkonstruovat takový fuzzy automat, který rozpoznává odpovídající jazyk odpovídající jazyku rozpoznávaném naším bivalentním automatem. 

Důležité je zmínit, že nelze zkonstruovat fuzzy automat, rozpoznávající stejný jazyk neboť fuzzy automat rozpoznává fuzzy jazyk, zatímco bivaletní automat klasický \uv{bivalentní} jazyk. Můžeme však sestavit automat takový, který přijímá řetězce ve stupni $0$ nebo $1$ podle toho, jestli je přijímal bivaletní automat.

Formálně řečeno, pro konečný (nedeterministický) bivalentní automat $A$ budeme konstruovat nedeterministický fuzzy automat $A'$ takový, že bude pro všechna $x \in \Sigma^*$ splněna následující rovnost:
$$
 \mathcal{L}(A')(\omega) = 
  \begin{cases}
   1 & \text{ pokud } \omega \in \mathcal{L}(A) \\
   0 & \text{ pokud } \omega \notin \mathcal{L}(A)
  \end{cases}
$$

\begin{note}
 Postup budeme provádět pro nedeterministické automaty. To jednak proto, že nedeterministické automaty jsou obecnější, než deterministické, a navíc, protože jsou v praxi využívány častěji. \TODO{ozdrojovat, klidně někde, kde rozebírám determinizmus vs. nedeterminizmus}
\end{note}

Nyní se podíváme na to, jak výsledný fuzzy automat bude vypadat. Abeceda i množina stavů automatu zůstanou zachovány, lišit se tedy bude pouze množina počátečních a koncových stavů a přechodová funkce. \TODO{fuzzy subset $I$, $F$ a $\delta$? nebo tak něco, z teorie fuzzy množin?}

\begin{definition}[Fuzzy automat bivalentního automatu] \label{def-FuzzAutBivAut}
  Mějme řetězec konečný nedeterministický automatu $A = ( Q, \Sigma, \delta, I, F )$. Pak nedetermintistický fuzzy automat přijímájící korespondující jazyk je automat $A' = ( Q, \Sigma, \mu, \sigma, \epsilon )$ kde pro všechna $q_i, q_j \in Q$ a $x \in \Sigma$:
  \begin{itemize}
   \item $\sigma(q_i) =
    \begin{cases}
     1 & \text{ pokud } q_i \in I \\
     0 & \text{ pokud } q_i \notin I
    \end{cases}$
    
   \item $\eta(q_i) =
    \begin{cases}
     1 & \text{ pokud } q_i \in F \\
     0 & \text{ pokud } q_i \notin F
    \end{cases}$
    
    \item $\mu(q_i, x, q_j) = 
     \begin{cases}
      1 & \text{ pokud } q_j \in \delta(q_i, x) \\
      0 & \text{ pokud } q_j \notin \delta(q_i, x)
     \end{cases}$
  \end{itemize}
\end{definition}

\TODO{rozebrat, jestli tento automat skutečně dělá to, co má? Asi by to chtělo}

\TODO{vymyslet nějaký pěkný příklad}

%%%%%%%%%%%%%%%%%%%%%%%%%%%%%%%%%%%%%%%%%%%%%%%%%%%%%%%%%%%%%%%%%%%%%%%%%%%%%%%
%%%%%%%%%%%%%%%%%%%%%%%%%%%%%%%%%%%%%%%%%%%%%%%%%%%%%%%%%%%%%%%%%%%%%%%%%%%%%%%
\section{Rozpoznávání textových vzorů}

Rozpoznávání vzorů obecně je jednou z nejvýznamějších aplikací informatiky. V běžném životě se často setkáváme se situacemi, kdy je třeba v datech najít výskyt učitého vzoru, popř. jeho další vlastnosti. Případně určit podobnost ke vzoru, nebo nejpodobnější vzor.

Typickým příkladem je např. detekce obličeje na fotografii, tedy rozpoznávání vzorů v obrazových datech. Vzory je však možné rozpoznávat v téměř jakýchkoliv datech, například textech, zvukových záznamech či výsedcích měření nebo pozorování.

Z pohledu teoretické informatiky je však základem vyhledávání vzorů v textových datech. Textová data, tedy řetězce, mají jednoduchou strukturu a lze s nimi snadno manipulovat. Na druhou stranu, jsou schopna reprezentovat nebo kódovat široké spektrum dat. Právě z tohoto důvodu je studium rozpoznávání textových vzorů klíčové pro zpracovávání jakýchkoliv dalších typů dat.

\begin{note}
 Pokud nebude uvedeno jinak, pojem \uv{rozpoznávání textových vzorů} bude v této kapitole zkracován jen na \uv{rozpoznávání vzorů}.
\end{note}

%%%%%%%%%%%%%%%%%%%%%%%%%%%%%%%%%%%%%%%%%%%%%%%%%%%%%%%%%%%%%%%%%%%%%%%%%%%%%%%
\subsection{Formální zavedení problému}
Stejně tak, jak se mohou různit aplikace rozpoznávání vzorů, i samotný pojem \uv{rozpoznávání vzorů} bývá chápan různě. V nejzákladnější podobě se jedná o problém určení, zda-li pozorovaný řetězec odpovídá předem stanovenému vzoru. Vzorem bývá obvykle také řetězec, ale může jím být například regulérní výraz. Také - může nás zajímat buď exaktní shoda pozorovaného řetězce se vzorem, nebo jen nějaká forma podobnosti. 

V rozšířeném smyslu může být problém chápán jako klasifikace. Tedy, určení třídy, do které by měl pozorovaný řetězec spadat, typicky na základě podobnosti s vybrannými reprezentanty jednotlivých tříd. 

V této kapitole se však budeme zabývat pouze určováním podobnosti vzorového a pozorovaného řetězce. U každé instance problému budeme znát abecedu se kterou pracujeme a také vzor. Vzorem bude libovolný řetězec nad touto abecedou. Řešením tohoto problému pro nějaký, tzv. pozorovaný, vstupní řetězec bude úroveň podobnosti tohoto řetězce s vzorovým. Jako podobnost zde budeme uvažovat reálné číslo z intervalu $[0, 1]$, kde $0$ znamená úplnou rozdílnost a $1$ úplnou shodu. 

\begin{note}
 Vzorový řetězec budeme v této kapitole vždy značit $\omega$, pozorovaný pak $\alpha$.
\end{note}

Nyní máme zadefinován problém samotný, nicméně je třeba zdůraznit, že v jeho definici se používá vágní pojem \uv{podobnost řetězců}. Podobnost řetězců je totiž pojem, který souvisí s konkrétní instancí problému a nelze jej nějak přesně, ale současně dostatečně obecně popsat. Jedinné, co o podobnosti řetězců můžeme říct, je, že čím vyšší toto číslo je, tím by si měly být řetězce podobnější.

Například, budeme-li porovnávat vstup zadaný z klávesnice počítače oproti nějakému vzoru, je možné, že uživatel udělá překlep. V takovém případě bude vzorovému řetězci určitě více podobný řetězec obsahující dva překlepy (záměna symbolu za některý sousedící na klávesnici) než jiný, který se sice bude lišit jen v jednom symbolu, ale to takovém, který je na opačné straně klávesnice.

Obdobně, pokud budeme pracovat s abecedou malých a velkých písmen (majuskule a minuskule). Uvažujme vzorový řetězec \str{hello}. Řetězec \str{HELLO} se s ním neshoduje v ani jednom symbolu, ale přesto jejich podobnost může být blízká jedné.

% \begin{definition}[Problém rozpoznávání vzorů]
%   Mějme konečnou abecedu $\Sigma$ a řetězce $\omega, \alpha \in \Sigma^*$ nad touto abecedou. Řetězec $\omega$ nazývejme \term{vzor}, řetězec $\alpha$ pak \term{pozorovaný řetězec}. Problém rozpoznávání vzorů je problém určení podobnosti řetězce $\alpha$ od řetězce $\omega$. Pokud nebude uvedeno jinak, podobnost uvažijeme jako reálné číslo z intervalu $[0, 1]$.
% \end{definition}

%%%%%%%%%%%%%%%%%%%%%%%%%%%%%%%%%%%%%%%%%%%%%%%%%%%%%%%%%%%%%%%%%%%%%%%%%%%%%%%
\subsection{Motivace k použití fuzzy automatů}
Klasická teorie automatů vznikla jako nástroj pro zpracování textových řetězců. Z tohoto důvodu je rozpoznávání textových vzorů jejím základním výsledkem. Automaty obecně jsou nástroje sloužící pro rozhodování, zda-li řetězec odpovídá vzoru automatem reprezentovaném. Použití pro rozpoznávání řetězcového vzoru tak bude jen speciálním případem jejich užití.

V předchozí podkapitole jsme si stanovili, že řešením našeho problému je číslo z intervalu $[0,1]$. Z tohoto důvodu nebude možné využít klasické bivalentní automaty. Fuzzy automaty pracují se stupněm pravdivosti, který by mohl s hodnotou podobnosti řetězců korespondovat. Navíc, v praxi se často setkáme s texty, které jsou nepřesné a nedokonalé. Fuzzy přístup by nám tak mohl pomoci na tyto nepřesnosti adekvátně reagovat.

\TODO{a co pravděpodobnostní?}
%%%%%%%%%%%%%%%%%%%%%%%%%%%%%%%%%%%%%%%%%%%%%%%%%%%%%%%%%%%%%%%%%%%%%%%%%%%%%%%
\subsection{Automat rozpoznávající $\omega$}
Klíčovým pro rozpoznávání vzorů (chceme-li využívat fuzzy automaty) je bivalentní automat rozpoznávající vzorový řetězec. Tedy automat takový, který přijímá jedinný řetězec $\omega$ a všechny ostatní zamítá. Nyní si takovýto automat zkonstruujeme.

Uvažujme, že máme k dispozici vzorový řetězec $\omega$ nad abecedou $\Sigma$. Označme $\mathcal{L}(\omega)$ jako jednoprvkový jazyk obsahující pouze řetězec $\omega$. Vzhledem k tomu, že jazyk $\mathcal{L}(\omega)$ je konečný, je také regulérní a existuje tak konečný deterministický automat, který jej rozpoznává.

Automat bude v každém kroku konzumovat symboly ze vstupního řetězce a porovnávat je se symboly vzorového řetězce na odpovídajících pozicích. Pokud dojde ke shodě na všech pozicích, automat dojde do koncového stavu a sledovaný řetězec přijme. Pokud se symboly shodovat nebudou, automat nebude mít definován žádný odpovídající přechod, kterým by pokračoval ve výpočtu, a řetězec tak zamítne. 

Takovýto automat označme jako \term{automat rozpoznávající} $\omega$.
\begin{definition}[Automat rozpoznávající $\omega$ (deterministický)]
  Mějme řetězec $\omega$ délky $n$ nad abecedou $\Sigma$. Automat rozpoznávající $\omega$ je pak konečný automat $A(\omega) = ( Q, \Sigma, \delta, q_0, F )$ takový, že jeho množina stavů $Q$ se sestává z právě $n$ stavů $q_0, \dots, q_n$, $q_0$ je počáteční stav, $F = \{ q_n \}$ množina koncových stavů a $\delta$ je přechodová funkce definována pro všechna $0 \leq k < n$ následovně:
  $$
    \delta(q_{k}, a_k) = q_{k+1} \text{ kde $a_k$ je $k$-tý symbol řetězce $\omega$}
  $$
\end{definition}

Tato definice automatu je vcelku intuitivní. K stejnému výsledku bychom došli, pokud bychom automat zkonstruovali konverzí gramatiky nebo regulérního výrazu. 

\begin{example}
 Příklad automatu rozpoznávající řetězec $\omega = $ \str{hello} se nachází na obrázku \ref{diag-AutRozpHell}.  

  \begin{figure}
    \includegraphics{diags0}
    \caption{Automat rozpoznávající \str{hello}} \label{diag-AutRozpHell}
  \end{figure}
\end{example}

My však budeme potřebovat fuzzy automat rozpoznávající $\omega$. To znamená, že musíme nejdříve automat z předchozí definice převést na nedeterministický a poté dle definice \ref{def-FuzzAutBivAut} k němu zkonstruovat odpovídající fuzzy automat.

\begin{definition}[Automat rozpoznávající $\omega$ (nedeterministický)] \label{def-AutRozpOme}
  Mějme řetězec $\omega$ nad abecedou $\Sigma$ z předchozí definice. Nedeterministický automat rozpoznávající $\omega$ je pak konečný automat $A'(\omega) = ( Q, \Sigma, \delta, I, F )$ takový, že jeho množina stavů $Q$ je stejná jako v předchozí definici, dále $I = \{ q_0 \}$ je množina počátečních a $F = \{ q_n \}$ množina koncových stavů a $\delta$ je přechodová funkce definována pro všechna $0 \leq k < n$ následovně:
  $$
  \delta(q_{k}, a_k) = 
  \begin{cases}
      \{ q_{k+1} \}	&\text{pokud je $a_k$ $k$-tý symbol řetězce $\omega$} \\
      \emptyset	&\text{jinak}
  \end{cases}
  $$
\end{definition}

%TODO: Ekvivalentní definice je zavedena v \cite[str. 2]{AstGariGonVillFar-ApprStrMatUsiDefFuzzAutLearExpr}.
Následuje vytvoření fuzzy automatu.

\begin{definition}[Fuzzy automat rozpoznávající $\omega$] \label{def-FuzzAutRozpOme}
  Mějme řetězec $\omega$ nad abecedou $\Sigma$ délky $n$. Fuzzy automat rozpoznávající $\omega$ je pak automat $A''(\omega)$ vytvořený z nedeterministického automatu rozpoznávající $\omega$ (definice \ref{def-AutRozpOme}) dle definice \ref{def-FuzzAutBivAut}. Bude to tedy automat $A''(\omega) = ( Q, \Sigma, \mu, \sigma, \epsilon )$ kde
  \begin{itemize}
   \item $\sigma(q_0) = 1$ a $\sigma(q_i) = 0$ pro všechna $i > 0$
   \item $\epsilon(q_n) = 1$ a $\epsilon(q_i) = 0$ pro všechna $i < n$
   \item $\mu(q_{k}, a_k, q_{k+1}) = 
      \begin{cases}
	1		&\text{pokud je $a_k$ $k$-tý symbol řetězce $\omega$} \\
	0		&\text{jinak}
      \end{cases}
      $
  \end{itemize}
\end{definition}

Nyní máme k dispozici fuzzy automat, který ostře rozpoznává vzorový řetězec. V následujících podkapitolách následuje výčet několika technik, které tuto ostrost (pomocí dalších informací) odstraňují a nahrazují podobností.

%%%%%%%%%%%%%%%%%%%%%%%%%%%%%%%%%%%%%%%%%%%%%%%%%%%%%%%%%%%%%%%%%%%%%%%%%%%%%%%
\subsection{Podobnost symbolů}
Nejzákladnější technika pro zanesení neostrého (stupňovitého) rozpoznávání je s využitím podobnostní relace symbolů. Tato technika byla přejata z \cite{RamGir-ConvFinAutFuzzAutStrComp}. Myšlenkou této techniky je, že symbol v pozorovaném řetězci může být snadno zaměněn za jiný, podobný, jemu odpovídající v řětězci vzorovém.

Pro realizaci této techniky je potřeba mít k dispozici fuzzy relaci $g_s: \Sigma \times \Sigma \rightarrow [0, 1]$. Tato relace popisuje podobnost dvojice symbolů. Tedy, je-li pro nějakou dvojici symbolů $x, y \in \Sigma$  $g_s(x, y) = 0$, pak se jedná o naprosto rozdílné symboly. Naopak, pokud bude $g_s(x, y) = 1$, pak se jedná o shodné symboly. Je zjevné, že by relace $g_s$ měla být symetrickou a reflexivní. \TODO{v článku to nepíší, ale měla by to být relace ekvivalence (Sym, Ref, Tra). Existuje něco, jako fuzzy relace ekvivalence?}

\begin{example}
 Jako příklad podobnostní relace (nad abecedou písmen anglické abecedy) může posloužit například vzdálenost patřičných kláves na klávesnici. V takovém případě by určitě platilo kupříkladu $g_s(a,s) > g_s(a, d) > g_s(a, l)$. Protože klávesy \str{A} a \str{S} jsou si blíž (a tudíž symboly $a$ a $s$ jsou si \uv{podobnější}) než například \str{A} a \str{D} či \str{A} a \str{L}.
 
 Jiným příkladem může být například vizuální podobnost napsaných (malých psacích) písmen. V takovém případě by zřejmě platilo $g_s(a, o) > g_s(m, t)$, protože malá psací písmena \str{a} a \str{o} jsou si vizuálně podobnější než \str{m} a \str{t}, která vypadají úplně rozdílně.
\end{example}

Máme-li k dispozici relaci $g_s$, je nutné ji zakomponovat do automatu. Jak autoři uvádějí, tato technika může pracovat s libovolným konečným automatem. Podíváme se proto nejdříve, jak využít relaci $g_s$ obecně. Následně ji aplikujeme na automat rozpoznávající $\omega$, čímž získáme nástroj pro podobnostní rozpoznávání textového vzoru.

\begin{definition}[Automat pracující s $g_s$] \label{def-AutPracGS}
Uvažujme, že máme nedeterminstický automat $A$ a relaci podobnosti symbolů $g_s$. Pak k automatu $A$ můžeme zkonstruovat fuzzy automat $A'$, který navíc pracuje s $g_s$. Takový automat bude zkonstruován dle definice \ref{def-FuzzAutBivAut} s tím rozdílem, že přechodová funkce $\mu$ bude definována pro všechna $q_i, q_j \in Q$ a $x \in \Sigma$ následovně:
$$
 \mu(q_i, x, q_j) = \bigvee_{y \in \Sigma} (g_s (x, y) \wedge \delta_y (q_i, q_j))
$$
 kde $\delta_x(q_i, q_j) = 
  \begin{cases}
   1 & \text{ pokud } q_j \in \delta(q_i, x)		\\
   0 & \text{ pokud } q_j \notin \delta(q_i, x)
  \end{cases}
 $ pro všechna $q_i, q_j \in Q$ a $x \in \Sigma$.
\end{definition}

Definice je vcelku přímočará. Pro každý přechod ze stavu $q_i$ do stavu $q_j$ přes symbol $x$, procházíme přechody původního automatu. Obsahovala-li přechodová funkce původního automatu přechod ze stavu $q_i$ přes symbol $y$ do stavu $q_j$, pak je $g_s (x, y) \wedge \delta_y (q_i, q_j)$ rovno podobnosti $x$ a $y$. V opačném případě je roven nule. Hodnota tohoto výrazu je díky spojení přes všechny symboly maximalizována.

Nyní aplikujeme tento způsob konstrukce fuzzy automatu na automat rozpoznávající $\omega$.

\begin{definition}[Automat rozpoznávající $\omega$ pracující s $g_s$]
 Mějme abecedu $\Sigma$, řetězec $\omega$ nad touto abecedou a fuzzy relaci $g_s$ nad touto abecedou. Dle definice \ref{def-AutRozpOme} můžeme zkonstruovat nedeterministický bivalentní automat $A(\omega)$ rozpoznávající $\omega$. Jako automat rozpoznávající $\omega$ pracující s $g_s$ označme nedeterministický fuzzy automat $A'(\omega)$, který byl z automatu $A(\omega)$ vytvořen podle definice \ref{def-AutPracGS}.
\end{definition}

\TODO{neměl by se takovýto automat místo $A'(\omega)$ značit třeba $A_{g_s}(\omega)$? }

Následuje jednoduchý příklad takového automatu.

\begin{example} \label{ex-AutRozpOmePodSym}
 Mějme abecedu $\Sigma = \{ a, b, c, d \}$. Dále uvažujme relaci podobnosti symbolů $g_s$ takovou, že
 \begin{itemize}
  \item každý symbol je podobný sám sobě ve stupni $1$
  \item každý symbol je podobný symbolu ve stupni $0,5$ jedná-li se o symboly reprezentující sousedící písmena abecedy
  \item každý symbol je podobný symbolu ve stupni $0,3$ jedná-li se o symboly reprezentující ob-jedno písmeno sousedící písmena abecedy
  \item všechny ostatní dvojice symbolů jsou si podobny ve stupni $0$
 \end{itemize}
 Tuto relaci můžeme zapsat do matice (sloupce i řádky odpovídají po řadě symbolům $a, b, c, d$):
 $$
 g_s = 
  \begin{pmatrix}
   1,0 	& 0,5	& 0,3	& 0,0 	\\
   0,5	& 1,0 	& 0,5	& 0,3	\\
   0,3	& 0,5	& 1,0 	& 0,5	\\
   0,0 	& 0,3	& 0,5	& 1,0 	\\
  \end{pmatrix}
 $$
 Nyní mějme vzorový řetězec $\omega = abc$. Pak můžeme podle předchozí definice sestavit automat $A(\omega)$ rozpoznávající $\omega$ pracující s $g_s$. Přechodový diagram takového automatu je na obrázku \ref{diag-AutRozpABCPracGS}.
 
 Tento automat evidentně rozpoznává řetězec \str{abc} ve stupni $1$. Pokud v pozorovaném řetězci nahradíme symbol \str{a} za \str{b}, bude jej automat přijímat ve stupni $0,5$. Pokud nahradíme \str{b} za \str{d}, bude jej automat přijímat ve stupni $0,3$.
 
 Pokud na začátek pozorovaného řetězce vložíme symbol \str{a} (tedy $\alpha = $ \str{aabc}), automat jej přijme ve stupni $0$. Stejnětak, pokud odebereme symbol \str{c} z konce vzorového řetězce (tedy $\alpha = $ \str{ab}). Pokud vložíme symbol \str{a} na začátek a současně odebereme \str{c} z konce vzorového řetězce, obdržíme pozorovaný řetězec $\alpha = $ \str{aab}. Tento řetězec bude přijat ve stupni $0,5$ \TODO{$1 \otimes 0,5 \otimes 0,5$, záleží tedy na $\otimes$}.
 
  \begin{figure}
    \includegraphics{diags5}
    \caption{Automat rozpoznávající \str{abc} pracující s $g_s$} \label{diag-AutRozpABCPracGS}
  \end{figure}
\end{example}

Z příkladu jasně vyplývá, že automat pracující s $g_s$ je schopen akceptovat pouze náhradu symbolu jiným symbolem. Bude-li pozorovaný řetězec oproti vzorovému obsahovat vložený symbol nebo naopak z něj bude symbol odebrán, tento typ automatu selže. Na druhou stranu jeho princip i konstrukce jsou jednoduché a snadno se s nimi pracuje.

%%%%%%%%%%%%%%%%%%%%%%%%%%%%%%%%%%%%%%%%%%%%%%%%%%%%%%%%%%%%%%%%%%%%%%%%%%%%%%%
\subsection{Fuzzy symboly}
Fuzzy symbol je technika využjívající podobnosti symbolů. Ve své podstatě se jedná o téže techniku jak v předchozí podkapitole, jen je na ni nahlíženo jinak. Oproti podobnosti symbolů je použití fuzzy symbolů komplikovanější, avšak umožňuje jednoduše tuto techniku kombinovat s jinými. Princip fuzzy symbolů byl přejat z \cite{Ech-DefSyssConPre}.

Mějme abecedu $\Sigma$ a relaci $p$ podobnosti symbolů (stejně jako relace $g_s$ v předchozí podkapitole). Fuzzy symbolem symbolu $x \in \Sigma$ označujeme fuzzy množinu symbolů takových, které jsou podle relace $p$ symbolu $x$ \uv{podobné}.

\begin{definition}[Fuzzy symbol]
Mějme abecedu $\Sigma$ a fuzzy relaci $p \subseteq \Sigma \times \Sigma$. Pak pro každý symbol $y \in \Sigma$ definujeme fuzzy symbol $\widetilde{y}$ symbolu $y$ jako fuzzy množinu nad $\Sigma$ takovou, že pro všechna $x \in \Sigma$ platí
$$
\widetilde{y}(x) = p(y, x)
$$
\end{definition}

Vzhledem k tomu, že fuzzy symbol máme definován pro všechny $y \in \Sigma$, můžeme množinu všech takových fuzzy symbolů nazvat abecedou fuzzy symbolů.

\begin{definition}[Abeceda fuzzy symbolů]
Mějme abecedu $\Sigma$ a fuzzy symboly $\widetilde{y}$ pro všechna $y \in \Sigma$. Pak množinu všech těchto fuzzy symbolů nazvěme abeceda fuzzy symbolů abecedy $\Sigma$ a označme $\widetilde{\Sigma}$. Tedy $\widetilde{\Sigma} = \{ \widetilde{y} \mid y \in \Sigma \}$.
\end{definition}

Máme-li abecedu fuzzy symbolů $\widetilde{\Sigma}$, můžeme pracovat s řetězci $\widetilde{\alpha} \in \widetilde{\Sigma}^*$ nad touto abecedou. Ještě si však doplníme, jak vytvořit k řetězci $\alpha \in \Sigma^*$ jemu odpovídající řetězec fuzzy symbolů $\widetilde{\alpha} \in \widetilde{\Sigma}^*$.

\TODO{sjednotit značení $\omega$ vs. $\alpha$, když se používá jen jeden obecný řetězec}

\begin{definition}[Řetězec fuzzy symbolů]
Mějme abecedu $\Sigma$ a nějaký řetězec $a_1 \dots a_n = \alpha \in \Sigma^*$. Pak definujme $\widetilde{\alpha} = \widetilde{a_1} \dots \widetilde{a_n}$ jako řetězec fuzzy symbolů řetězce $\alpha$.
\end{definition}

V této fázi jsme schopni plnohodnotně pracovat s řetězci fuzzy symbolů a konstruovat je z řetězců nad abecedou $\Sigma$. Nyní přejdeme k návrhu fuzzy automatu, který bude s fuzzy symboly pracovat. Stejně jako u podobnosti symbolů i fuzzy symboly mohou být aplikovány na libovolný typ automatu. Vytvoříme proto automat pracující s fuzzy symboly nejdříve obecně, pro libovolný automat $A$. 

\begin{definition}[Automat pracující s fuzzy symboly] \label{def-AutPracFuzzSym}
  Mějme nedeterministický fuzzy automat automat $A$. Pak fuzzy automat $\widetilde{A}$ pracující s fuzzy symboly vytvoříme tak, že v definici automatu $A$ nahradíme $\Sigma$ za $\widetilde{\Sigma}$.
  
  \TODO{může to tak být? a co související pojmy}
\end{definition}

Formální zavedení automatu pracujícího s fuzzy symboly je intutivní, jedná se jen o formalitu. Abychom však využili potenciál fuzzy symbolů, je třeba pozměnit výpočet automatu. Proces jeho výpočtu se změní ve fázi výpočtu přechodové funkce fuzzy stavů. Připomeňme, že ta je definována (definice \ref{def-PreFunFuzzStav}) jako fuzzy relace $\widehat{\mu}$ přiřazující každému fuzzy stavu $V$ a fuzzy symbolu $x$ fuzzy stav dle předpisu
$$
 \widehat{\mu}(V, x) = V \circ \mu[x]
$$

Zde je zjevně nutné nahradit $\mu[x]$ spojením přes všechny fuzzy symboly. Bude tedy vypadat následovně:
$$
 \widehat{\mu}(V, x) = V \circ \bigvee_{y \in \Sigma} (\mu[x] \wedge \widetilde{x}(y))
$$

Tím, že je změna zakořeněna ve výpočtu automatu, nám umožňuje další práci se samotným automatem. Můžeme tedy bez problémů zkonstruovat automat rozpoznávající $\omega$ pracující s fuzzy symboly.

\begin{definition}[Automat rozpoznávající $\omega$ pracující s fuzzy symboly]
 Mějme abecedu $\Sigma$, řetězec $\omega$ nad touto abecedou a abecedu fuzzy symbolů $\widetilde{\Sigma}$. Dle definice \ref{def-FuzzAutRozpOme} můžeme zkonstruovat fuzzy automat $A(\omega)$ rozpoznávající $\omega$. Následně pak podle definice \ref{def-AutPracFuzzSym} automat $\widetilde{A}(\omega)$ rozpoznávající $\omega$ pracující s fuzzy symboly.
\end{definition}

Postup konstrukce takovéhoto automatu je opět vcelku intuitivní. Následuje demonstrace na příkladu.

\begin{example}
 Mějme abecedu $\Sigma$, vzorový řetězec a podobnostní relaci $p = g_s$ stejné jako v příkladu \ref{ex-AutRozpOmePodSym}. Na obrázku \ref{diag-AutRozpOmePraFuzSym} je zobrazen diagram automatu $\widetilde{A}(\omega)$ rozpoznávající $\omega$ pracující s fuzzy symboly.
 
 \begin{figure}
    \includegraphics{diags6}
    \caption{Automat rozpoznávající \str{abc} pracující s $g_s$} \label{diag-AutRozpOmePraFuzSym}
  \end{figure}
\end{example}

Co se týče vlastností automatů (rozpoznávajících $\omega$) pracujících s fuzzy symboly, jejich charakteristika je vesměs stejná jako u automatů pracujících s podobností symbolů. Pouze, jak již bylo zmíněno v úvodu, nezasahují do struktury automatu jako takového.

%%%%%%%%%%%%%%%%%%%%%%%%%%%%%%%%%%%%%%%%%%%%%%%%%%%%%%%%%%%%%%%%%%%%%%%%%%%%%%%
\subsection{Editační operace}
Další technikou pro podobnostní porovnávání pozorovaného a vzorového řetězce je s využitím editačních operací. Tato technika byla přejata z \cite{AstGonMenGar-FuzzAutEpsMovCmpFuzzMeasBtwStrs}. Základní idea této techniky spočívá v trojici jednoduchých editačních operací, jejiž složením jsme schopni popsat transformaci pozorovaného řetězce na vzorový. Množství transformace pak udává podobnost pozorovaného a vzorového řetězce.

Následuje formální definice editačních operací a pojmů s nimi souvisejících. Následně přejdeme ke konstrukci automatu, který s nimi bude schopen pracovat.

\begin{definition}[Editační operace]
Mějme abecedu $\Sigma$, uvažujme množinu $E = (\Sigma \cup \{ \epsilon \}) \times (\Sigma \cup \{ \epsilon \}) \setminus \{ (\epsilon, \epsilon) \}$. Pak každou dvojici $(x, y) = z \in E$ nazvěme editační operace. Speciálně pak, pro všechna $x, y \in \Sigma$, $(x, y) \in E$ znamená nahrazení symbolu $x$ symbolem $y$, $(x, \epsilon) \in E$ znamená odebrání symbolu $x$ a naopak $(\epsilon, y) \in E$ pak vložení symbolu $y$. Navíc jako editační operaci uvažujme i všechny dvojce $(x, x) \in E$ (pro každé $x \in \Sigma$) symbolizující \uv{žádnou editaci}.

Máme-li editační operaci $(x, y) = z \in E$, pak označme $x = z^\downarrow$ a $y = z^\uparrow$.
\end{definition}

Editační operace jsou tedy tři a to náhrada symbolu, vložení symbolu a odebrání symbolu. Například řetězec \str{hallo} vznikl záměnou \str{e} za \str{a} v řetězci \str{hello}. Obdobně, řetězec \str{hellow} vznikl přidáním \str{w} na konec a řetězec \str{helo} odebráním (prvního nebo druhého) symbolu \str{l}.

My však obvykle očekáváme, že došlo k více, než jedné jednoduché editaci. Je proto vhodné zavést koncept mnohanásobné editace. Jednotlivé editace za sebe seřadíme do posloupnosti v pořadí, v jakém mají být postupně aplikovány, a takovouto posloupnost nazvěme vyrovnáním řetězce $\alpha$ na řetězec $\omega$.

Uvažujme nyní množinu $E$ editačních operací jako abecedu. Pak každé vyrovnání $\zeta$ řetězce $\alpha$ na řetězec $\omega$ (posloupnost $z_1 z_2 \dots z_n$ symbolů $z_1, z_2, \dots z_n \in E$), tak můžeme považovat za řetězec nad abecedou $E$.

\TODO{fakt E považovat za abecedu a $G$ za jazyk? není to zbytečná komplikace? je to tam nutné?}

Například všechny tři následující řetězce jsou vyrovnáním řetězce \str{ahoj} na řetězec \str{hello}:
\begin{align*}
 \zeta_1 =& (a,\epsilon) (h,h) (o, e) (j, l) (\epsilon, l) (\epsilon, o) \\
 \zeta_2 =& (a, \epsilon) (h, \epsilon) (o, \epsilon) (j, \epsilon) (\epsilon, h) (\epsilon, e) (\epsilon, l) (\epsilon, l) (\epsilon, o) \\
 \zeta_3 =& (a,h) (h,e) (o,l) (j,l) (\epsilon, o) 
\end{align*}

Na tomto příkladu je vhodné si povšimnout, že obecně může existovat více než $1$ vyrovnaní mezi libovolnou dvojicí řetězců. Bude proto vhodné neuvažovat vyrovnání jednotlivá, ale množinu všech možných vyrovnání mezi dvojicí řetězců. 

Podíváme-li se nyní jen na levé části editačních operací ve vyrovnání $\zeta_1$ z předchozího příkladu, zjistíme, že jejich zřetězením získáme řetězec $\alpha$:
$$
(a,\epsilon)^\downarrow (h,h)^\downarrow  (o, e)^\downarrow  (j, l)^\downarrow  (\epsilon, l)^\downarrow  (\epsilon, o)^\downarrow  = ahoj
$$
Stejně tak, zřetězením pravých částí editačních operací v $\zeta_1$ získáme řetězec $\omega$:
$$
(a,\epsilon)^\uparrow (h,h)^\uparrow  (o, e)^\uparrow  (j, l)^\uparrow  (\epsilon, l)^\uparrow  (\epsilon, o)^\uparrow  = hello
$$

Tato vlastnost nám udává, v jakém pořadí mají být editační operace aplikovány. Stejně tak nám odstaňuje nadbytečné editační operace (např. opakované přidávání a odebírání téže znaku, které by mohlo vést až k nekonečné posloupnosti editací). Proto nám tato vlastnost poslouží jako definiční pro formání zavedení vyrovnání řetězců.

\begin{definition}[Vyrovnání řetězců \cite{AstGonMenGar-FuzzAutEpsMovCmpFuzzMeasBtwStrs}]
 Jako množinu všech vyrovnání $G(\alpha, \omega)$ řetězce $\alpha$ na řetězec $\omega$ (kde $(\epsilon, \epsilon) \neq \alpha, \omega \in \Sigma^*$) označme takovou množinu $\{ \zeta \in E^+ \mid \zeta \text{splňuje vlastnosti 1., 2. i 3.}\}$
 \begin{enumerate}
  \item $\zeta = z_1 z_2 \dots z_r$, $z_i \neq (\epsilon, \epsilon)$ pro všechna $i \in 1, \dots, r$,
  \item ${z_1}^\downarrow {z_2}^\downarrow \dots {z_r}^\downarrow = \alpha$
  \item ${z_1}^\uparrow {z_2}^\uparrow \dots {z_r}^\uparrow = \omega$
 \end{enumerate}
\end{definition}

V tento okamžik máme formálně zavedena vyrovnání řetězců. Můžeme tedy přejít k práci s nimi. Ukážeme si způsob, jak pomocí vyrovnání řetězců spočítat podobnost dvojice řetězců. Na základě tohoto výpočtu pak sestavíme automat, který tento výpočet bude realizovat.

Pro uřčení podobnosti na základě vyrovnání řetězců budeme potřebovat znát míry pravdivosti editačních operací. Vstupem pro výpočet podobnosti řetězců tak bude navíc binární fuzzy relace $R$ nad množinou všech editačních operací ($E$), udávající stupeň akceptovatelnosti kadždé z možných editačních operací. S touto znalostí můžeme nadefinovat relaci podobnosti řetězců, tzv. fuzzy míru dvojice řetězců $\alpha$ a $\omega$.

\TODO{Musí být $R$ reflexivní a symetrická (def. automatu to vyžaduje, ale je to nutné?). A co T-tranzitivita?}

\begin{definition}[Fuzzy míra\cite{AstGonMenGar-FuzzAutEpsMovCmpFuzzMeasBtwStrs}]\label{def-FuzzMir}
 Mějme binární fuzzy relaci $R$ nad $\Sigma \cup \{\epsilon\}$. Pak jako fuzzy míru mezi řetězci $\alpha, \omega \in \Sigma^*$ (značenou $S_{\Sigma, R, \otimes}$) označme fuzzy relaci danou následujícím předpisem:
 $$
 S_{\Sigma, R, \otimes} = 
  \begin{cases}
   1	&	\text{ pokud } (\alpha, \omega) = (\epsilon, \epsilon) \\
   \max\limits_{\zeta \in G(\alpha, \omega)} (\bigotimes_{i = 1}^{|\zeta|} R(\zeta_i) )	&	\text{ pokud } (\alpha, \omega) \neq (\epsilon, \epsilon) 
  \end{cases}
 $$
% kde $\bigotimes_{i = 1}^{|\zeta|} R(\zeta_i) = R(\zeta_1) \otimes R(\zeta_2) \otimes \dots \otimes R(\zeta_{|\zeta|}$ pro $zeta_1 zeta_2 \dots zeta_{|\zeta|} = \zeta \in G(\alpha, \omega)$.
\end{definition}

Definice fuzzy míry je vcelku intuitivní. Počítá se míra všech možných vyrovnání z nichž se vybírá ta největší. Míra vyrovnání se určuje jako t-norma ze všech $R(\zeta_i)$, tedy stupňů akceptovatelnosti jednotlivých editačních operací. Navíc, míra mezi dvojicí prázdných řetězců je dodefinována jako $1$.

Označme $\mathcal{L}(\omega)$ jako fuzzy jazyk řetězců \uv{podobných} řetězci $\omega$ s podobností danou relací $R$. Takový jazyk pak můžeme nadefinovat pro  všechna $\alpha \in \Sigma^*$ následujícím předpisem
$$
 \mathcal{L}(\omega)(\alpha) = S_{\Sigma, R, \otimes}(\alpha, \omega)
$$

Nyní zkonstruujeme nedeterministický fuzzy automat s $\epsilon$-přechody, který jazyk $\mathcal{L}$ rozpoznává. 
\TODO{rozpoznává vs. přijímá, pozor na to}

\begin{definition}[Automat rozpoznávající $\mathcal{L}$ \cite{AstGonMenGar-FuzzAutEpsMovCmpFuzzMeasBtwStrs}] \label{def-AutRozpCalL}
 Mějme binární fuzzy relaci $R$ nad $\Sigma \cup \{\epsilon\}$ (stejná jako v definici \ref{def-FuzzMir}). Pak pro vzorový řetězec $a_1 a_2 \dots a_n = \omega \in \Sigma^*$ označme $M_{\Sigma, R, \otimes}(\omega)$ automat rozpoznávající jazyk $\mathcal{L}(\omega)$ dle definice \ref{def-NedFuzzAutEpsPre}, takový, že
 \begin{enumerate}
  \item množina stavů $Q = \{ q_0, q_1, \dots q_n \}$
 % \item $\Sigma$ automatu je stejná jako $\Sigma$ v parametru
  \item fuzzy přechodová funkce $\mu$ pro všechny $x \in \Sigma$:
  \begin{enumerate}
   \item $\mu(q_i, q_i, x) = R(x, \epsilon)$ pro všechny $q_i \in Q$ taková, že $i = 0, \dots, n$
   \item $\mu(q_i, q_{i+1}, x) = R(x, a_{i+1})$ pro všechny $q_i, q_{i+1} \in Q$ taková, že $i = 0, \dots, n-1$
   \item $\mu(q, q', x) = 0$ pro všechny $q, q' \in Q$ nesplňující předchozí dva body

   \item $\mu(q_i, q_i, \epsilon) = 1$ pro všechny $q_i \in Q$ taková, že $i = 0, \dots, n$
   \item $\mu(q_i, q_{i+1}, \epsilon) = R(\epsilon, a_{i+1})$ pro všechny $q_i \in Q$ taková, že $i = 0, \dots, n-1$
   \item $\mu(q, q', \epsilon) = 0$ pro všechny $q, q' \in Q$ nesplňující předchozí dva body 
  \end{enumerate}
  \item množina počátečních stavů $\sigma$: $\sigma(q_0) = 1$ a pro všechny ostatní $q_0 \neq q' \in Q$: $\sigma(q') = 0$ 
  \item množina koncových stavů $\eta$: $\eta(q_n) = 1$ a pro všechny ostatní $q_n \neq q' \in Q$: $\eta(q') = 0$
 \end{enumerate}
\end{definition}

Máme nadefinován fuzzy automat rozpoznávající jazyk $\mathcal{L}(\omega)$. Bylo by však vhodné dokázat, že jazyk $\mathcal{L}(\omega)$, který tento automat rozpoznává je skutečně jazykem řetězců podobných řetězci $\omega$ s podobností danou relací $R$. Vzhledem ke složitosti důkazu tohoto tvrzení se v této práci spokojíme pouze s ilustrací na příkladu.
\begin{theorem}\label{the-JazAutJeJazEditOper}
 Mějme binární fuzzy relaci $R$ nad $\Sigma \cup \{\epsilon\}$ (stejná jako v definici \ref{def-FuzzMir}) a vzorový řetězec $\omega \in \Sigma^*$. Pak pro automat $M_{\Sigma, R, \otimes}(\omega)$ sestavený dle předcházející definice a fuzzy míru $S_{\Sigma, R, \otimes}$ platí následující rovnost
 $$
  \mathcal{L}(M_{\Sigma, R, \otimes}) = \mathcal{L}(\omega)
 $$
\end{theorem}
\begin{proof}
 Kompletní důkaz je k nalezení v \cite{AstGonMenGar-FuzzAutEpsMovCmpFuzzMeasBtwStrs}.
\end{proof}

\TODO{sazba symbolů v matematickém módu vs. verbatim řetězce v textovém}

\begin{example} \label{ex-AutRozpCalL}
 Uvažujme abecedu $\Sigma = \{a, b, c\}$ a vzorový řetězec $\omega = abc$. Zkonstruujeme automat, který bude akceptovat ve stupni $0.5$ náhradu symbolu $x$ symbolem s ním v abecedě sousedícím. Navíc uvažujme vložení symbolu $a$ ve stupni $0.2$ a odebrání symbolu $c$ ve stupni $0.1$. Tedy, relace $R$ bude vypadat následovně:
 \begin{align*}
  R = \{ 
   & (a, a)/1, (b, b)/1, (c, c)/1, \\
   & (b, a)/0.5, (a, b)/0.5, (c, b)/0.5, (b, c)/0.5, \\
   & (a, \epsilon)/0.2, (\epsilon, c)/0.1 \}
 \end{align*}

 Dle definice \ref{def-AutRozpCalL} můžeme sestavit automat $M_{\Sigma, R, \otimes}$. Jako $\otimes$ použijme produktovou t-normu. Získáme tak automat $M_{\Sigma, R, \otimes} = (Q, \Sigma, \mu, \sigma, \epsilon)$ takový, že:
  \begin{enumerate}
  \item $Q = \{ q_0, q_1, q_2, q_3 \}$
  \item $\mu = \{ $
  \begin{enumerate}
   
   \item $(q_0, q_0, a)/0.2, (q_1, q_1, a)/0.2, (q_2, q_2, a)/0.2, (q_3, q_3, a)/0.2),$
   \item $(q_0, q_1, a)/1, (q_1, q_2, a)/0.5,$ \\
   $(q_0, q_1, b)/0.5, (q_1, q_2, b)/1, (q_2, q_3, b)/0.5,$ \\
   $(q_1, q_2, c)/0.5, (q_2, q_3, c)/1,$
   \addtocounter{enumii}{1} %\item  \TODO{takové, jako $(q_3, q_1, a)$}
   \item $(q_0, q_0, \epsilon)/1, (q_1, q_1, \epsilon)/1, (q_2, q_2, \epsilon)/1, (q_3, q_3, \epsilon)/1),$
   \item $(q_2, q_3, \epsilon)/0.1$
   \addtocounter{enumii}{1} %\item \TODO{takové, jako $(q_3, q_1, \epsilon)$}
  \end{enumerate}
  $ \} $ (přechody dle bodů (b) a (d) v definici jsou s nulovým stupněm a ve výpisu jsou vynechány)
  \item $\sigma = \{ q_0/1, q_1/0, q_2/0, q_3/0  \}$
  \item $\eta = \{ q_0/0, q_1/0, q_2/0, q_3/1  \}$
  \\ \TODO{$\sigma = \{ x / y, ... \}$ by se mělo přepsat na $\sigma(x) = y, ...$, ne?}
 \end{enumerate}

 Přechodový diagram tohoto automatu je k nalezení na obrázku \ref{img-AutRozpCalL}. V přechodovém diagramu jsou červeně zvárazněny pravidla pro rozpoznávání $\omega$, ostatní pravidla (doplněna dle definice) jsou černá. 
 
 Nyní si na pár řetězcích zkuzme ukázat platnost věty \ref{the-JazAutJeJazEditOper}. Dle definice \ref{def-RetPriAut} spočítáme stupeň, v jakém automat náš testovací rozpoznává řetězec $\alpha$:
 \begin{align*}
  M_{\Sigma, R, \otimes}(\alpha) 
   =& \max_{q \in Q} (\mu^*(\sigma, \alpha)(q) \otimes \eta(q))) \\
   =& \max\{ \mu^*(\sigma, \alpha)(q_0) \otimes \eta(q_0)), \mu^*(\sigma, \alpha)(q_1) \otimes \eta(q_1)),  \\
    & \mu^*(\sigma, \alpha)(q_2) \otimes \eta(q_2)), \mu^*(\sigma, \alpha)(q_3) \otimes \eta(q_3)) \}   \\
   =& \max\{ \mu^*(\sigma, \alpha)(q_0) \otimes 0), \mu^*(\sigma, \alpha)(q_1) \otimes 0),  \\
    & \mu^*(\sigma, \alpha)(q_2) \otimes 0), \mu^*(\sigma, \alpha)(q_3) \otimes 1) \}   \\
   =&  \mu^*(\sigma, \alpha)(q_3)
 \end{align*}
  
 \begin{itemize}
  \item řětězec $\alpha = abc$:
  Určíme stupeň akceptance automatem:
  $$
  M_{\Sigma, R, \otimes}(\alpha) 
   = \mu^*(\sigma, abc) (q_3)
   = \widehat{\mu}(\widehat{\mu}(\widehat{\mu}(\widehat{\mu}(\sigma, \epsilon), a), b), c) (q_3)
   = \{ q_3 / 1 \} (q_3) 
   = 1 
  $$
  A následně ověříme fuzzy míru. Množina všech vyrovnání bude obsahovat například $\zeta_1 = (a,a), (b,b), (c,c)$ či 
  $\zeta_2 = (a, \epsilon) (\epsilon, a) (b, \epsilon) (\epsilon, b) (c, \epsilon) (\epsilon, c)$. Snadno zjistíme, že $\bigotimes_{i = 1}^{|\zeta|} R(\zeta_i) $ je maximální právě pro $\zeta = \zeta_1$ a nabývá stupně $1$. A tedy $S_{\Sigma, R, \otimes}(\alpha) = 1$.
   
  \item řětězec $\alpha = ab$:
  $$
  M_{\Sigma, R, \otimes}(\alpha) 
   = \mu^*(\sigma, ab) (q_3)
   = \widehat{\mu}(\widehat{\mu}(\widehat{\mu}(\sigma, \epsilon), a), b) (q_3)
   = \{ q_2 / 1, q_3 / 0.1 \} (q_3)
   = 0,1
  $$
  $$
  S_{\Sigma, R, \otimes}(\alpha) 
   = R(a, a) \otimes R(b, b) \otimes R(\epsilon, c) 
   = 1 \otimes 1 \otimes 0.1
   = 0,1
  $$
  
  \item řětězec $\alpha = bbb$:
  $$
  M_{\Sigma, R, \otimes}(\alpha) 
   = \mu^*(\sigma, bbb) (q_3)
   = ...
   = \{ q_3 / 0,25 \} (q_3)
   = 0,25
  $$
  $$
  S_{\Sigma, R, \otimes}(\alpha) 
   = R(b, a) \otimes R(b, b) \otimes R(b, c) 
   = 0,5 \otimes 1 \otimes 0,5
   = 0,25
  $$
  
  \item řětězec $\alpha = abaca$:
  $$
  M_{\Sigma, R, \otimes}(\alpha) 
   = \mu^*(\sigma, abaca) (q_3)
   = ...
   = \{ q_3 / 0,04 \} (q_3)
   = 0,04
  $$
  $$
  S_{\Sigma, R, \otimes}(\alpha) 
   = R(a, a) \otimes R(b, b) \otimes R(a, \epsilon) \otimes R(c, c) \otimes R(a, \epsilon)
   = 1 \otimes 1 \otimes 0,2 \otimes 1 \otimes 0,2
   = 0,04
  $$

  \item řětězec $\alpha = cba$:
  $$
  M_{\Sigma, R, \otimes}(\alpha) 
   = \mu^*(\sigma, cba) (q_3)
   = ...
   = (\emptyset) (q_3)
   = 0
  $$
  $$
  S_{\Sigma, R, \otimes}(\alpha) 
   = R(c, a) \otimes R(b, b) \otimes R(a, c) 
   = 0 \otimes 1 \otimes 0
   = 0
  $$
  
  ...
  
 \end{itemize}

 Je tedy zjevné, že výpočet automatu $M_{\Sigma, R, \otimes}$ je v korespondenci s fuzzy mírou $S_{\Sigma, R, \otimes}$.
 
 \begin{figure}
  \includegraphics{diags4}
  \caption{Přechodový diagram automatu z příkladu \ref{ex-AutRozpCalL}} \label{img-AutRozpCalL}
 \end{figure}
\end{example}

Je vidět, že automat zkonstruován dle editačních operací je značně silný nástroj. Umožňuje nám velmi pohodlně popsat, jak moc mohou být konkrétní editační operace akceptovány. Editační operace vložení symbolu, náhrada symbolu a odebrání symbolu jsou pro popis modifikace vzorového řetězce přirozené.

Nevýhodou této techniky je, že jednotlivé editační operace jsou akceptovány bez ohledu na jejich výskyt v řetězci. Automat akceptuje nastanuvší editační operaci pokaždé, kde může nastat, ve stejném stupni. Často je však třeba v jiném stupni stejnou editační operaci přijímat např. na začátku a na konci řetězce pokaždé však v jiném stupni. Tento požadavek však automat zkonstruovaný pomocí editačních operací neumí zpracovat. Řešením může být například použití následující techniky.


%%%%%%%%%%%%%%%%%%%%%%%%%%%%%%%%%%%%%%%%%%%%%%%%%%%%%%%%%%%%%%%%%%%%%%%%%%%%%%%
\subsection{Deformovaný automat}
Dalším ze způsobů, jak přijímat řetězec podobný vzorovému je s využitím deformovaného (fuzzy) automatu. Tato technika využívá tzv. deformovaného automatu, neboli automatu který byl stanoveným způsobem upraven, neboli deformován. Tato technika byla přejata z \cite{AstGariGonVillFar-ApprStrMatUsiDefFuzzAutLearExpr}.

Jako deformace může být použita prakticky jakákoliv úprava automatu. Mějme fuzzy automat $A$. Provedením deformace $x$ získáme deformovaný automat $A'$, který rozpoznává jiný jazyk, než automat $A$. Mezi nejzákladnější tři deformace patří náhrada symbolu, vložení symbolu před symbol a odebrání symbolu. Možných deformací existuje nekonečně mnoho. Mezi další deformace může patřit například (pro nějaké $x, y, z \in \Sigma$ a $i \geq 0$): \uv{náhrada symbolu $x$ na $i$-té pozici symboly $yz$}, \uv{odebrání všech výskytů symbolu $x$, které se nachází před symbolem $y$} nebo \uv{vložení sudého počtu symbolů $y$ mezi symboly $x$ a $z$}.

Provedeme-li deformaci fuzzy automatu rozpoznávající $\omega$, můžeme se na trojici základních deformací podívat konkrétně. Ukázka toho, jak by vypadal deformovaný automat po provedení jedné ze základních deformací je vyobrazeno v tabulce \ref{tbl-DefAutDef}.

\begin{table}[h]
 \begin{tabular}{|l|l|}
  \hline
  Deformace	& Význam deformace	 \\
  \hline
  \textsc{Náhrada} symbolu na $i$-té pozici (symbolu $x$) symbolem $y$ & \includegraphics{diags1} \\	  
    $\delta' = \delta \cup \{ (q_i, y, q_{i+1}) \}$, $Q' = Q$	&	\\
  \textsc{Vložení} symbolu $y$ na $i$-tou pozici (před symbol $x$) & \includegraphics{diags2} \\
    $\delta' = \delta \setminus \{ (q_i, x, q_{i+1} ) \} \cup \{ (q_i, \epsilon, q'_i), (q_i, y, q'_i), (q'_i, x, q_{i+1}) \}$, $Q' = Q \cup \{ q'_i \}$	 & \\
  \textsc{Odebrání} symbolu z $i$-té pozice (symbolu $x$) & \includegraphics{diags3} \\
    $\delta' = \delta \cup (q_i, \epsilon, q_{i+1})$, $Q' = Q$	 & \\
  \hline
 \end{tabular}
 \caption{Deformace deformovaného automatu}\label{tbl-DefAutDef}
 \TODO{pozor, toto je pro konečné automaty, ne pro fuzzy automaty!}
\end{table}

Je vidět, že deformace mohou být účinným nástrojem pro rozpoznávání modifikovaných pozorovaných řetězců. Na druhou stranu, deformování automatu vyžaduje znalost fungování automatů. Často také může nastat situace, kdy výsledný zdeformovaný automat bude více, než deformovaný automat rozpoznávající $\omega$, automatem reprezentující samostatný netriviální vzor.

\subsection{Shrnutí}
V této kapitole byl zaveden pojem rozpoznávání textových vzorů. Bylo ukázáno, že pro klasickou teorii automatů je to triviální problém, který však kvůli nepřesnostem reálných dat vyžaduje nasazení fuzzy automatů. Bylo představeno několik technik, které pomocí fuzzy automatů umožňují přijímání řetězců podobných vzorovému. 

Nejjednodušší z nich, využívající relaci podobnosti symbolů, je vhodná na prosté nahrazování podobných symbolů. Technika fuzzy symbolů funguje na stejném principu, jen se liší ve formálním zavedení. Technika s využitím editačních operací umožňuje specifikovat stupeň akceptance základních editačních operací (vložení symbolu, odebrání symbolu, náhrada symbolu). Poslední technika, deformovaný automat, umožňuje libovolnou transformaci automatu, vedoucí až k libovolnému vzoru.


%%%%%%%%%%%%%%%%%%%%%%%%%%%%%%%%%%%%%%%%%%%%%%%%%%%%%%%%%%%%%%%%%%%%%%%%%%%%%%%
%%%%%%%%%%%%%%%%%%%%%%%%%%%%%%%%%%%%%%%%%%%%%%%%%%%%%%%%%%%%%%%%%%%%%%%%%%%%%%%
%%%%%%%%%%%%%%%%%%%%%%%%%%%%%%%%%%%%%%%%%%%%%%%%%%%%%%%%%%%%%%%%%%%%%%%%%%%%%%%
\newpage
\bibliography{resources}
\bibliographystyle{plain}


\end{document}
