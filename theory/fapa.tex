\documentclass[a4paper,10pt]{article}
\usepackage[utf8x]{inputenc}
\usepackage[czech]{babel}

\usepackage{amsthm}
\usepackage{amsmath}
\usepackage{amsfonts}
\usepackage{mathtools}

%% číslovaný cases
\usepackage[subnum]{cases}

%% colspan a rowspan v tabulkách
\usepackage{multirow}

%%  http://tex.stackexchange.com/a/361157
\usepackage{showlabels}

%% subfigures, https://en.wikibooks.org/wiki/LaTeX/Floats,_Figures_and_Captions#Subfloats
\usepackage{subcaption}

%% číslování řádků, https://texblog.org/2012/02/08/adding-line-numbers-to-documents/
\usepackage{lineno}

%Definice, věta, důkaz!
\newtheorem{definition}{Definice}
\newtheorem{theorem}{Věta}
\newtheorem*{notation*}{Značení}
\newtheorem{note}{Poznámka}
\newtheorem{example}{Příklad}


\numberwithin{definition}{section}
\numberwithin{theorem}{section}
%\numberwithin{notation}{section}
\numberwithin{note}{section}
\numberwithin{example}{section}


%Ostatní makra
\newcommand{\TODO}[1]{ \textit{\small(zde bude doplněno: #1)} }	
\newcommand{\RLY}{?? }

\newcommand{\lattL}{\mathbb{L}}
\newcommand{\fsubsets}[1]{\mathcal{F}(#1)}

\newcommand{\term}[1]{\textit{#1}}
\newcommand{\str}[1]{{\ttfamily#1}}
\newcommand{\regex}[1]{{\ttfamily#1}}

\newcommand{\impl}[2]{{\ttfamily#1/}{\ttfamily test/data/}{\ttfamily#2}}

% Speciální nastavení

% vycentrované floating objekty
% http://tex.stackexchange.com/questions/2651/should-i-use-center-or-centering-for-figures-and-tables
\makeatletter
\g@addto@macro\@floatboxreset\centering
\makeatother

\linenumbers
\modulolinenumbers[10]


%opening
\title{Aplikace fuzzy a pravděpodobnostních automatů}
\author{Martin Jašek}
\date{12. září 2016 --- ??}

\begin{document}

\maketitle
\tableofcontents

\newpage

\section{Vybrané aplikace, poznámky}

%%%%%%%%%%%%%%%%%%%%%%%%%%%%%%%%%%%%%%%%%%%%%%%%%%%%%%%%%%%%%%%%%%%%%%%%%%%%%%%
\subsection{Fuzzy jazyky, gramatiky a regulární výrazy}


\subsubsection*{Fuzzy finite automata and fuzzy regular expressions with membership values in lattice-ordered monoids \cite{LiPed-FuzzFinAutFuzzRegExMembValLattOrdMon}}

Fuzzy regulární výrazy, to je taky tak trochu aplikace, ne? Bohužel tento článek taky není k dispozici ...


\subsubsection*{Minimization of lattice finite automata and its application to the decomposition of lattice languages \cite{LiPed-MinLattFinAutAppDecoLattLang}}

Tady se asi trošku zabředne trochu víc do matematiky, ale -- uvidíme. Uvidíme, co vyleze z dekompozice fuzzy jazyků.

\begin{definition}
 \TODO{lattice language} ($\sim$ fuzzy jazyk nad svazem)?
\end{definition}


%%%%%%%%%%%%%%%%%%%%%%%%%%%%%%%%%%%%%%%%%%%%%%%%%%%%%%%%%%%%%%%%%%%%%%%%%%%%%%%
\subsection{Fuzzy regulátory a řídící systémy}

\subsubsection*{Fuzzy Reasoning and Fuzzy Automata in User Adaptive Systems \cite{Kov-ReasFuzzAutInUsrAdpSyss}}

Řídící systémy, adaptivní systémy? Každopádně, bude to něco na principu fuzzy regulátorů. Nejspíš to bude fungovat tak, že vstupní řetězec bude posloupnost real--time událostí vně systému. A on na ně bude nějak reagovat. A něco dělat.

\begin{definition}
 \TODO{Fuzzy reazoning}
 \TODO{Fuzzy state transition rulebase}
\end{definition}

V příkladu uvádí nástroj pro výběr nejlepší židle. Uživatelé jsou požádání o označení jak moc na ně působí zvolené židle (co do modernosti, pohodlnnosti, ceny, ...) a systém vygeneruje uspořádání židlí od \uv{nejlepší} po \uv{nejhorší}.

Víc toho nevím, protože text působí hodně odborně a vyžaduje tak znalost základních pojmů.


\subsubsection*{A model for Finite-state probabilistic systems \cite{BruFu-ModelFinStateProbSyss}}

Na tento článek se odkazuje \cite{MorMal-FuzzAutAndLangs} v kapitole 10.1. Ale kromě anotace jsem se o něm nedozvěděl nic víc.


%%%%%%%%%%%%%%%%%%%%%%%%%%%%%%%%%%%%%%%%%%%%%%%%%%%%%%%%%%%%%%%%%%%%%%%%%%%%%%%
\subsection{Pattern recognition}


\subsubsection*{Syntactic Methods in Pattern Recognition \cite{Fu-SynMethPattRec}}

Knížka (celá) o pattern recognizition pomocí formálních jazyků. Bohužel jen klasických automatů případně stochastických.



%%%%%%%%%%%%%%%%%%%%%%%%%%%%%%%%%%%%%%%%%%%%%%%%%%%%%%%%%%%%%%%%%%%%%%%%%%%%%%%
\subsection{Strojové učení, Fuzzy neuronové sítě}

\subsubsection*{Application of Fuzzy Automata Theory and Knowledge Based Neural Networks for Development of Basic Learning Model \cite{DarAhmSin-AppFuzzAutTheKnBsNeuNetDevBasLeaMod}}

Zkombinováním neuronové sítě a fuzzy automatu vznikla tzv. Fuzzy  Automata  based  Neural  Network (FANN). Prý znalostní pravidla převádí na fuzzy automaty a z nich poté konstruuje FANN.

\begin{definition}
 \TODO{Model of Learning}
 \TODO{Knowledge rule}
\end{definition}


Autoři FANN demonstrují na Urban Trafic modeling.

\begin{definition}[Urban Trafic modeling]
 je problém modelace infrastruktury, predikce budoucího vývoje. Inteligentní řízení provozu a podob.
\end{definition}

\subsubsection*{A Formulation of Fuzzy Automata and Its Application as a Model of Learning Systems \cite{WeeFu-FormFuzzzAutItsAppModLearSyss}}

Originální paper bohužel není k dispozici. Ale dle anotace je to jeden z prvních článků vůbec. Podobá se stochastickému automatu.

\begin{definition}
 \TODO{Stochastický automat}
\end{definition}

Autoři předvádějí použití \uv{fuzzy automatu} jako model strojového učení (používají zastaralý výraz \uv{Model of learning system}). Dále uvádějí hypotézu použití jako automatická kontrola a pattern recognizition \TODO{jak se to řekne česky?}.

Na konci zmiňují, že použití fuzzy automatu v těchto situacích přináší především jednoduchost (návrhu i průběhu výpočtu).


\subsubsection*{Fuzzy Neural Networks \cite{LeeLee-FuzzNeuNet}}

To bude nejspíš podobné jako \cite{DarAhmSin-AppFuzzAutTheKnBsNeuNetDevBasLeaMod}. 


%%%%%%%%%%%%%%%%%%%%%%%%%%%%%%%%%%%%%%%%%%%%%%%%%%%%%%%%%%%%%%%%%%%%%%%%%%%%%%%
\subsection{Bioinformatika a medicína}




\subsubsection*{An application of intuitionistic fuzzy sets in medical diagnosis \cite{SupBisAkh-AppIntFuzzSetMedDiag}}

Zní to příšerně, ale nejspíš to bude vycházet z \cite{GupSar-FuzzAutDecProc}.


%%%%%%%%%%%%%%%%%%%%%%%%%%%%%%%%%%%%%%%%%%%%%%%%%%%%%%%%%%%%%%%%%%%%%%%%%%%%%%%
\subsection{Ostatní aplikace a zdroje}

\subsubsection*{Fuzzy automata and decision processes\cite{GupSar-FuzzAutDecProc}}

To zní lákavě, ale bohužel online není prakticky ani anotace.


\subsubsection*{An introductory survey of fuzzy control ??}

Podívat se na tohle, jestli se tam vůbec mluví o použití (fuzzy) automatů:

http://www.sciencedirect.com/science/article/pii/002002558590026X


\subsubsection*{Determinism and fuzzy automata \cite{Bel-DetFuzzAut}}

Článek určitě zajímavý do teoretické části, ale dost možná se bude zmiňovat o (alespoň!) některých aplikacích.



%%%%%%%%%%%%%%%%%%%%%%%%%%%%%%%%%%%%%%%%%%%%%%%%%%%%%%%%%%%%%%%%%%%%%%%%%%%%%%%
%%%%%%%%%%%%%%%%%%%%%%%%%%%%%%%%%%%%%%%%%%%%%%%%%%%%%%%%%%%%%%%%%%%%%%%%%%%%%%%
\section{Jiná literatura}

\subsection{Další čtení}
\begin{itemize}
 \item Morderson, Malik: Fuzzy Automata and Languages: Theory and Applications \cite{MorMal-FuzzAutAndLangs}
 \item Introduction to Probabilistic Automata \cite{Paz-IntroProbAut}, samostatná kapitola se seznamem vybraných aplikací pravděpodobnostních automatů
 \item New directions in fuzzy automata \cite{DooKre-NewDirFuzzAut}: Nový pohled na automaty, v anotaci mluví o tom, jak dlouho se automaty již studují, takže by se tam mohl odkazovat na nějaké aplikace. Aplication driven methodology?
 \item ...
 \item ...
 \item ...
 
\end{itemize}

\subsection{Třeba by se mohlo hodit}
\begin{itemize}
 \item Social Cognitive Learning Theory and other Theories and Models, https://www.learning-theories.com/
 \item Zadeh: Fuzzy algorithms, http://www.sciencedirect.com/science/article/pii/S0019995868902118
\end{itemize}

\newpage
\bibliography{resources}
\bibliographystyle{plain}


\end{document}
