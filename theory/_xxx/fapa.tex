\documentclass[a4paper,10pt]{article}
\usepackage[utf8x]{inputenc}
\usepackage[czech]{babel}

\usepackage{amsthm}
\usepackage{amsmath}
\usepackage{amsfonts}
\usepackage{mathtools}

%% číslovaný cases
\usepackage[subnum]{cases}

%% colspan a rowspan v tabulkách
\usepackage{multirow}

%%  http://tex.stackexchange.com/a/361157
\usepackage{showlabels}

%% subfigures, https://en.wikibooks.org/wiki/LaTeX/Floats,_Figures_and_Captions#Subfloats
\usepackage{subcaption}

%% číslování řádků, https://texblog.org/2012/02/08/adding-line-numbers-to-documents/
\usepackage{lineno}

%Definice, věta, důkaz!
\newtheorem{definition}{Definice}
\newtheorem{theorem}{Věta}
\newtheorem*{notation*}{Značení}
\newtheorem{note}{Poznámka}
\newtheorem{example}{Příklad}


\numberwithin{definition}{section}
\numberwithin{theorem}{section}
%\numberwithin{notation}{section}
\numberwithin{note}{section}
\numberwithin{example}{section}


%Ostatní makra
\newcommand{\TODO}[1]{ \textit{\small(zde bude doplněno: #1)} }	
\newcommand{\RLY}{?? }

\newcommand{\lattL}{\mathbb{L}}
\newcommand{\fsubsets}[1]{\mathcal{F}(#1)}

\newcommand{\term}[1]{\textit{#1}}
\newcommand{\str}[1]{{\ttfamily#1}}
\newcommand{\regex}[1]{{\ttfamily#1}}

\newcommand{\impl}[2]{{\ttfamily#1/}{\ttfamily test/data/}{\ttfamily#2}}

% Speciální nastavení

% vycentrované floating objekty
% http://tex.stackexchange.com/questions/2651/should-i-use-center-or-centering-for-figures-and-tables
\makeatletter
\g@addto@macro\@floatboxreset\centering
\makeatother

\linenumbers
\modulolinenumbers[10]


%opening
\title{Aplikace fuzzy a pravděpodobnostních automatů}
\author{Martin Jašek}
\date{12. září 2016 --- ??}

\begin{document}

\maketitle
\tableofcontents
\newpage


%%%%%%%%%%%%%%%%%%%%%%%%%%%%%%%%%%%%%%%%%%%%%%%%%%%%%%%%%%%%%%%%%%%%%%%%%%%%%%%

\subsection{Ostatní důlěžité pojmy}
\subsubsection*{Nedeterministický fuzzy automat s $\epsilon$ přechody}

\begin{definition}[Nedeterministický fuzzy automat s $\epsilon$ přechody]\label{def-NedFuzzAutEpsPre}
\TODO{dohledat přesně, zkontrolovat a ozdrojovat}
 Nedeterministický fuzzy automat $A$ je pětice $(Q, \Sigma, \mu, \sigma, \eta)$, kde $Q$ je konečná množina stavů, $\Sigma$ je abeceda, $\mu$ je fuzzy přechodová funkce (fuzzy relace $Q \times (\Sigma \cup \{ \epsilon \}) \times Q \rightarrow [0, 1]$) a $\sigma$ a $\eta$ jsou po řadě fuzzy množiny nad $Q$ počátačních, resp. koncových stavů.
\end{definition}

\TODO{Tady by asi bylo vhodné rozebrat $\epsilon$-uzávěry}

\subsubsection*{Reprezentace fuzzy automatu}
\TODO{dohledat zdroje}

\TODO{Značení fuzzy množiny $\sigma = \{ x/0.5 \}$ vs. $\sigma(x) = 0.5$} 

Přechodový diagram: Notace s lomítky např. zde: \cite{GonGar-FuzzLangInfRanAccGuzzAutPumLemDetProc}.

Tabulka: stav $\times$ symbol nebo stav $\times$ stav \cite{DooKre-NewDirFuzzAut}?



\subsection{Další pojmy z úvodu do teorie automatů}
\subsubsection{Gramatiky}
Gramatiky (přesněji \uv{formální gramatiky}) jsou nástroje pro popis jazyků. Gramatika určuje, jakým způsobem lze jazyk vygenerovat. 
Následující pojmy jsou převzaty z \cite{Law-FinAut}.

\subsubsection*{Gramatika}
\begin{definition}[Formální gramatika]
 Jako formální gramatiku označujeme čtveřici $(N, T, P, S)$, kde
 \begin{enumerate}
  \item $N$ je abeceda tzv. neterminálních symbolů
  \item $T$ je abeceda tzv. terminálních symbolů ($T \cap N = \emptyset$)
  \item $P$ je množina přechodových pravidel, kde každé pravidlo je ve tvaru $x \rightarrow y$, $x, y \in (N \cup T)^*$ a $x$ obsahuje alespoň jeden neterminál
  \item $S$ je počáteční neterminál ($S \in N$)
 \end{enumerate}
\end{definition}

Mějme gramatiku $G = (N, T, P, S)$ takovou, že $X \in N$, $y \in T$ a $X \rightarrow y \in P$. Pak pro libovolné $u, v \in (N \cup T)^*$ o řetězci $uyv$ říkáme, že vznikl přímým odvozením z řetězce $uXv$ (píšeme $uXv \Rightarrow uyv$). Posloupnost $w_1 \Rightarrow \dots \Rightarrow w_n$ řetězců $w_1, \dots, w_n \in (N \cup T)^*$ zapisujeme $w_1 \Rightarrow^* w_n$ a nazýváme derivace.

Jako jazyk generovaný gramatikou $G$ značíme následující množinu:
$$
  L(G) = \{ w \in T^* | S \Rightarrow^* w \} 
$$

\begin{example}
 Mějme gramatiku $G = (N, T, P, S)$, kde $N = \{ M \}$, $T = \{ a, b \}$, $P = \{$ $S \rightarrow a M $, $M \rightarrow S b$, $M \rightarrow b$ $\}$. Pak platí $S \Rightarrow aM \Rightarrow a Sb \Rightarrow a aM b \Rightarrow a ab b$ a $L(G) = \{ a^n b^n | n \geq 1 \}$.
\end{example}

\subsubsection*{Regulérní gramatika}
Regulérní gramatika je speciálním případem gramatiky. Tento typ gramatiky má zásadní vliv pro studium automatů.

\begin{definition}[Regulérní gramatika]
 Gramatika $G = (N, T, P, S)$ se nazývá regulérní, pokud každé z jejich pravidel je ve tvaru $X \Rightarrow a Y$ nebo $X \Rightarrow a$, kde $X, Y \in N$ a $a \in T$ a dále platí, že $S$ se nevyskytuje na pravé straně žádného pravidla.
\end{definition}

Jazyk generovaný regulérní gramatikou se nazývá regulérní jazyk.

\begin{example}
 Regulérní gramatikou je například gramatika $G = (N, T, P, S)$, kde $N = \{ R \}$, $T = \{ a, b \}$, $P = \{$ $S \rightarrow a R $, $R \rightarrow a R$, $R \rightarrow b R$, $R \rightarrow a$ $\}$.
\end{example}

Regulérní gramatiky (a jim opdovídající jazyky) mají spoustu zajímavých vlastností. Uvedeme zde pouze jednu důležitou vlastnost.
\begin{theorem}\label{the:FinRegLang}
 Každý konečný jazyk $L$ je regulérním jazykem.
\end{theorem}
\begin{proof}
 Má-li být jazyk $L$ regulérním pak pro něj musí existovat gramatika $G$, která jej generuje. Podrobnější důkaz lze najít v literatuře.
\end{proof}

%%%%%%%%%%%%%%%%%%%%%%%%%%%%%%%%%%%%%%%%%%%%%%%%%%%%%%%%%%%%%%%%%%%%%%%%%%%%%%%
%%%%%%%%%%%%%%%%%%%%%%%%%%%%%%%%%%%%%%%%%%%%%%%%%%%%%%%%%%%%%%%%%%%%%%%%%%%%%%%
%%%%%%%%%%%%%%%%%%%%%%%%%%%%%%%%%%%%%%%%%%%%%%%%%%%%%%%%%%%%%%%%%%%%%%%%%%%%%%%
%%%%%%%%%%%%%%%%%%%%%%%%%%%%%%%%%%%%%%%%%%%%%%%%%%%%%%%%%%%%%%%%%%%%%%%%%%%%%%%
%%%%%%%%%%%%%%%%%%%%%%%%%%%%%%%%%%%%%%%%%%%%%%%%%%%%%%%%%%%%%%%%%%%%%%%%%%%%%%%
%%%%%%%%%%%%%%%%%%%%%%%%%%%%%%%%%%%%%%%%%%%%%%%%%%%%%%%%%%%%%%%%%%%%%%%%%%%%%%%


%%%%%%%%%%%%%%%%%%%%%%%%%%%%%%%%%%%%%%%%%%%%%%%%%%%%%%%%%%%%%%%%%%%%%%%%%%%%%%%

\section{Fuzzy automaty, jazyky, gramatiky a regulérní výrazy}
\TODO{nějak to uvést. Budou pojmy jako regulérní jazyk a gramatika popsány v nějaké předchozí kapitole?}

\TODO{pojem \uv{Lattice language}}

\TODO{značení: \uv{Fuzzy množina $\phi$} vs. \uv{$L$-množina $\phi: X \rightarrow L$}; \uv{fuzzy podmnožina} vs. \uv{fuzzy množina nad}}
% asi k ničemu: DooKre-NewDirFuzzAut \cite{DooKre-NewDirFuzzAut}, 
% asi zbytečně vědátorské, nám stačí jazyk nad [0,1]: PalYau-FracFuzzGramAppPatRec \cite{PalYau-FracFuzzGramAppPatRec}?,
% okay: GonGar-FuzzLangInfRanAccGuzzAutPumLemDetProc \cite{GonGar-FuzzLangInfRanAccGuzzAutPumLemDetProc},
% fjuuh: StaCir-ConsFuzzAutFuzzRegExp \cite{StaCir-ConsFuzzAutFuzzRegExp},
% blbost, nic tam není: RamGir-CharFuzzRegLang \cite{RamGir-CharFuzzRegLang},


\subsection{Jazyk rozpoznávaný fuzzy automatem}
Věta 6.3 \cite{LiPed-FuzzFinAutFuzzRegExMembValLattOrdMon} (pro lattice monoid, není to někde jen pro $[0,1]$?).

Dle definice 4 \cite{GonGar-FuzzLangInfRanAccGuzzAutPumLemDetProc} je fuzzy regulární jazyk fuzzy podmnožina bivalentního.

Automat s bivalentní $\mu$ and $\eta$ (a fuzzy $\sigma$) taky rozpoznává fuzzy regulérní jazyk \cite{PalYau-FracFuzzGramAppPatRec}. Neřešil něco takového i Bel? Jinak řečeno, support konečný automat \cite{GonGar-FuzzLangInfRanAccGuzzAutPumLemDetProc}.

Dále např. \cite{Pee-FuzAcSynPatRec}.

\subsection{Fuzzy a bivalentní regulérní jazyky}
Univerzum fuzzy jazyka je regulérní jazyk, pozorování 6.1 \cite{LiPed-FuzzFinAutFuzzRegExMembValLattOrdMon}.

Stejně tak, zaříznutý jazyk ($a$-řez jazyka) je také regulérní, věta 2.2 \cite{PalYau-FracFuzzGramAppPatRec}. 

Pumping lemma pro fuzzy regulérní jazyky, lemma 4-7 pro různé typy automatů \cite{GonGar-FuzzLangInfRanAccGuzzAutPumLemDetProc}.

Uzávěrové vlastnosti fuzzy regulérních jazyků, např. \cite{PalYau-FracFuzzGramAppPatRec}.
\subsection{Fuzzy regulární výrazy}
LiPed-FuzzFinAutFuzzRegExMembValLattOrdMon, definice 5.1, 5.2 (+ opsat důkaz, že $[0, 1]$ je lattice monoid) \cite{LiPed-FuzzFinAutFuzzRegExMembValLattOrdMon}.

Algoritmus převodu reg na aut, \cite{StaCir-ConsFuzzAutFuzzRegExp}. Ale zdá se mi to až moc složité.

%%%%%%%%%%%%%%%%%%%%%%%%%%%%%%%%%%%%%%%%%%%%%%%%%%%%%%%%%%%%%%%%%%%%%%%%%%%%%%%
%%%%%%%%%%%%%%%%%%%%%%%%%%%%%%%%%%%%%%%%%%%%%%%%%%%%%%%%%%%%%%%%%%%%%%%%%%%%%%%




%%%%%%%%%%%%%%%%%%%%%%%%%%%%%%%%%%%%%%%%%%%%%%%%%%%%%%%%%%%%%%%%%%%%%%%%%%%%%%%
%%%%%%%%%%%%%%%%%%%%%%%%%%%%%%%%%%%%%%%%%%%%%%%%%%%%%%%%%%%%%%%%%%%%%%%%%%%%%%%
\newpage
\section{Fuzzy tree automaty}
V následujících podkapitolách budou fuzzy tree automaty demonstrovány na konkrétních příkladech.

\subsection{Použití fuzy tree automatů}
Fuzzy tree automaty je možné použít všude tam, kde je třeba rozpoznávat určitým způsobem stromově strukturovaná data. Konečnost množiny $\mathcal{Q}$, pro kterou je definována přechodová funkce $\mu_X$ neterminálů $X \in N$ však přináší omezení na aritu každého uzlu, která tak vždy musí být konečným číslem.

Konečnost množiny $\mathcal{Q}$ však teoreticky nemusí být nutná. Teoreticky by šlo jako vzory funkce $\mu_X$ namísto konkrétních řetězců nad $Q$ použít například regulérní výraz popisující celou třídu takových řetězců (například \regex{(q0 | q1)+}). To by však zkomplikovalo implementaci takového automatu a proto toto rozšíření nebude uvažováno.

Důležité však je, že automat umožňuje rozpoznávat rekurzivní stromy. Rekurze je dosaženo (opakovaným) přechodem ze stavu do téže stavu (v nenulovém stupni). 

%TODO 
% \subsection{Jednoduchá klasifikace zvířat}
% Autor práce přichází s využitím fuzzy tree automatů pro jednoduchou klasifikaci zvířat. Každé zvíře je charakterizováno prostředím, kde obvykle žije nebo je možné jej spatřit (voda, souš, vzduch), povrchem těla (kůže, šupiny, srst, peří), seznamem končetin (nohy, ruce, křídla, ploutve) a ocasem (žádný, krátký, dlouhý).
% 
% Jmenovaným vlastnostem byly přiřazeny neterminální symboly $Env, Surf, Ext, Tail$ a terminální symboly po řadě $water, land, air, leather, scales, hair, feather, leg, hand, wing, fin, short, long$. Zvíře jako celek je pak reprezentováno terminálem $Animal$, který má čtveřici potomků, a to právě $Env, Surf, Ext$ a volitelně $Tail$\TODO{další volitelné \uv{atributy}? umí plavat? ve slané/sladké vodě? loví? létá? ...}. Ukázka stromů reprezentující zvířata pes, pták, netopýr, tyranosaurus, ryba, žába, krokodýl, pavouk, (chlupatá tarantule/čmelák?), vydra/bobr, velryba/delfín, létající ryba je na obrázku \TODO{obrázek!!!}.

%%%%%%%%%%%%%%%%%%%%%%%%%%%%%%%%%%%%%%%%%%%%%%%%%%%%%%%%%%%%%%%%%%%%%%%%%%%%%%%
%%%%%%%%%%%%%%%%%%%%%%%%%%%%%%%%%%%%%%%%%%%%%%%%%%%%%%%%%%%%%%%%%%%%%%%%%%%%%%%
\section{Buněčné fuzzy automaty}
%%%%%%%%%%%%%%%%%%%%%%%%%%%%%%%%%%%%%%%%%%%%%%%%%%%%%%%%%%%%%%%%%%%%%%%%%%%%%%%
\subsection{Obecně k aplikacím}
Buněčné automaty (obecně) nacházejí široké uplatnění v rozličných oblastech. Dle \cite{Wol-NewKinSci} se s nimi lze setkat v matematice, informatice, fyzice, biologii, společenských vědách, např. filozofii a umění. Používají se například pro simulace fyzikálních dějů (např. difuze, tok tekutin), krystalizace, biologickým, urbanistickým, enviromentalistickým a geografickým simulacím či například ke generování fraktálů.

Co se buněčných fuzzy automatů týče, jejich aplikace nejsou tak rozšířené. Jedním z důvodů, proč tomu tak je, je velká podobnost bivalentních buněčných automatů a buněčných fuzzy automatů. Jak bylo uvedeno v předchozí podkapitole, $[0,1]$-buněčný fuzzy automat je jen speciálním případem klasického byvalentního. Obdobně, práce s fuzzy If-Then pravidly je vlastně jen speciální případ klasických If-Then pravidel.

Například \cite{WhiEngInj-UsConsCelAuHiResModUrbLanUsDyn} používá čísla z intervalu $[0,1]$ jako vstupní informaci. \cite{DiwPatGup-CelAutBasEdDetBraTum} používá automat podobný buněčnému automatu s fuzzy logikou, který navíc pracuje se stavy na intervalu $[0,1]$. Obdobně, \cite{PatMor-EdgDetTecFuzzLogCEllLeaAutFuzzImPro} používá buněčné automaty v kombinaci s určitou formou fuzzy logiky (avšak automat napovažují za buněčný fuzzy automat). V \cite{MofSadRezMey-CelEdDetComCelAuCelLeaAu} je použit automat pracující na intervalu $[0,255]$, který by šel snadnou modifikací konvertován na $[0,1]$-buněčný fuzzy automat. Automat v \cite{BatXie-CellCity} vykazuje určitou náhodnost a bylo by tak možné považovat jej za pravděpodobnostní.
  
Další varianty buněčných automatů a jejich uplanění jsou vyjmenovány v \cite{NayPatMah-SurTwDimCelAutAppImProc}. Ve všech případech se uvažují vždy dvoudimenzionální bunečné automaty.

Aplikace fuzzy automatů se dají rozdělit do dvou kategorií. Do první z nich spadají urbanistické simulace. Pomocí buněčných fuzzy automatů tak byly řešeny simulace hustoty dopravy \cite{Pla-FuzCelModTrafDatFus}, růstu mořských řas u pobřeží \cite{CheMyn-ModAlgBloDutCosWat+} či plánování elektrické rozvodné sítě. Zdaleka nezčastější však bylo nasazení buněčných fuzzy automatů na řešení problému městského růstu. Z tohoto důvodu bude tomuto problému věnována samostatná podkapitola.

Další oblastí, kde nalezly buněčné fuzzy automaty uplatnění je zpracování obrazu. Používají se například na zaostřování obrazů \cite{MarMeySol-HybMetGasDifModFuzCelAutImSha}\cite{ZhaLiZha-EdgDetImaBasFuzCelAut}, hledání hran \cite{NayPatMah-SurTwDimCelAutAppImProc}\cite{MarMeySol-HybMetGasDifModFuzCelAutImSha}, vyhledávání vzorů \cite{WanJiaZhoDu-ImProcBasFuzCelAuMod}\cite{MajCha-FuzCelAutModPatClas} či odstranění šumu \cite{SadRetKam-EfMetImpNoiRedImFuzCelAut}\cite{SahUguSah-SalPepNoiFilFuzCelAut}\cite{NayPatMah-SurTwDimCelAutAppImProc}. V následujcíích podkapitolách budou některé takové techniky rozebrány podrobněji.

Co se pravděpodobnostních buněčných automatů týče, ty nacházejí uplatnění všude tam, kde je třeba dodat náhodnost a nepravidelnost. Například pro problém simulace městského růstu\cite{War+-StoConCelModUrbGro} \cite{Wu-CalStoCelAutAppRurUrbLanConv}. Dále pak například pro náhodné generátory, generátory šumu, simulace pohybu částic, difuze částic či nukleace (vznik krystalů) \cite{TofMar-CelAuMach+.pdf}.

Před studiem samotných aplikací je nutné dodat, že aplikace zde popsané jsou pouze teoretickým popisem sestaveným na základě použitých zdrojů. Každá z těchto aplikací vyžaduje kalibraci parametrů (např. počet generací, návrh fuzzy množin) a přizpůsobení na míru konkrétní instance problému. Většina zdrojů proto automaticky kombinuje několik technik dohromady. Například \cite{LiYe-NeuNetBasCelAutSim+} \cite{YehLi-SimLanDevIntCelAutMulCriEv} využívají buněčné fuzzy automaty v kombinaci s neuronovými sítěmi, v \cite{Ahm+-CalFuzCelAutModUrbDynSauAr} s genetickýni algoritmy a \cite{MarMeySol-HybMetGasDifModFuzCelAutImSha} \cite{MofSadRezMey-CelEdDetComCelAuCelLeaAu} s pomocí učícího se automatu. \TODO{dohledat, vysvětlit a ocitovat, co to je učící se automat} Většinou je také nutná hluboká znalost dané problematiky, nejlépe přítomnost experta.

%%%%%%%%%%%%%%%%%%%%%%%%%%%%%%%%%%%%%%%%%%%%%%%%%%%%%%%%%%%%%%%%%%%%%%%%%%%%%%%


%%%%%%%%%%%%%%%%%%%%%%%%%%%%%%%%%%%%%%%%%%%%%%%%%%%%%%%%%%%%%%%%%%%%%%%%%%%%%%%
\subsection{Zpracování obrazu}


%%%%%%%%%%%%%%%%%%%%%%%%%%%%%%%%%%%%%%%%%%%%%%%%%%%%%%%%%%%%%%%%%%%%%%%%%%%%%%%




%%%%%%%%%%%%%%%%%%%%%%%%%%%%%%%%%%%%%%%%%%%%%%%%%%%%%%%%%%%%%%%%%%%%%%%%%%%%%%%
%%%%%%%%%%%%%%%%%%%%%%%%%%%%%%%%%%%%%%%%%%%%%%%%%%%%%%%%%%%%%%%%%%%%%%%%%%%%%%%
\section{Strojové učení}


\subsection{Reálné příklady použití}




Autor sám přichází s myšlenkou, že kombinace fuzzy automatů a strojového učení má poměrně značný potenciál pro uplatnění v praxi. Kombinace fuzzy automatů, které poměrně jednoduše, avšak poměrně restriktivně popisují určitý jazyk mohou být pomocí strojového učení snadno upraveny na mocnější nástroje. Důvod, proč se však nepoužívají vidí v současném boomu neuronových sítí jako takových, které utlačují všechny ostatní techniky strojového učení.


%%%%%%%%%%%%%%%%%%%%%%%%%%%%%%%%%%%%%%%%%%%%%%%%%%%%%%%%%%%%%%%%%%%%%%%%%%%%%%%
\section{Biologie a medicína}

%%%%%%%%%%%%%%%%%%%%%%%%%%%%%%%%%%%%%%%%%%%%%%%%%%%%%%%%%%%%%%%%%%%%%%%%%%%%%%%
%%%%%%%%%%%%%%%%%%%%%%%%%%%%%%%%%%%%%%%%%%%%%%%%%%%%%%%%%%%%%%%%%%%%%%%%%%%%%%%
%%%%%%%%%%%%%%%%%%%%%%%%%%%%%%%%%%%%%%%%%%%%%%%%%%%%%%%%%%%%%%%%%%%%%%%%%%%%%%%
%%%%%%%%%%%%%%%%%%%%%%%%%%%%%%%%%%%%%%%%%%%%%%%%%%%%%%%%%%%%%%%%%%%%%%%%%%%%%%%
%%%%%%%%%%%%%%%%%%%%%%%%%%%%%%%%%%%%%%%%%%%%%%%%%%%%%%%%%%%%%%%%%%%%%%%%%%%%%%%

\newpage

\section{Další aplikace}


\subsection{Pattern matching, další příklady}







\TODO{zavést pojem ligvistická veličina (aka takový to fuzzy výrok/fuzzy množina)? L. A. Zadeh, “The concept of linguistic variable and its application to approximate reasoning”, Information sciences, 1975.}

\subsection{Ostatní a nezařezeno}


V \cite{SalKarMor-ModRobAtDeHuRoIntUsFiFuStMa} používají fuzzy automat pro vylepšení (zefektivnění) human-robot interakce. Nikde však neuvádějí, jak konkrétně.


\section{Pravděpodobností automaty}







\section{Konkrétní příklady}
\subsection{Automat jako \uv{lepší} reprezentace něčeho}
Markovův řetěz (probabilistický (dvoj)stavový systém).
Petriho sítě (petri nets).
Rozhodovací stromy.
Řídící systémy, událostní systémy (ale není to to samé, co petriho sítě?). \cite{GraPat-FuzLogFinStaMacModReaTimSys}



%%%%%%%%%%%%%%%%%%%%%%%%%%%%%%%%%%%%%%%%%%%%%%%%%%%%%%%%%%%%%%%%%%%%%%%%%%%%%%%
\newpage
\bibliography{resources}
\bibliographystyle{plain}


\end{document}

