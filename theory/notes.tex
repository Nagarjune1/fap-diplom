\documentclass[a4paper,10pt]{article}
\usepackage[utf8x]{inputenc}
\usepackage[czech]{babel}

\usepackage{amsthm}
\usepackage{amsmath}
\usepackage{amsfonts}
\usepackage{mathtools}

%% číslovaný cases
\usepackage[subnum]{cases}

%% colspan a rowspan v tabulkách
\usepackage{multirow}

%%  http://tex.stackexchange.com/a/361157
\usepackage{showlabels}

%% subfigures, https://en.wikibooks.org/wiki/LaTeX/Floats,_Figures_and_Captions#Subfloats
\usepackage{subcaption}

%% číslování řádků, https://texblog.org/2012/02/08/adding-line-numbers-to-documents/
\usepackage{lineno}

%Definice, věta, důkaz!
\newtheorem{definition}{Definice}
\newtheorem{theorem}{Věta}
\newtheorem*{notation*}{Značení}
\newtheorem{note}{Poznámka}
\newtheorem{example}{Příklad}


\numberwithin{definition}{section}
\numberwithin{theorem}{section}
%\numberwithin{notation}{section}
\numberwithin{note}{section}
\numberwithin{example}{section}


%Ostatní makra
\newcommand{\TODO}[1]{ \textit{\small(zde bude doplněno: #1)} }	
\newcommand{\RLY}{?? }

\newcommand{\lattL}{\mathbb{L}}
\newcommand{\fsubsets}[1]{\mathcal{F}(#1)}

\newcommand{\term}[1]{\textit{#1}}
\newcommand{\str}[1]{{\ttfamily#1}}
\newcommand{\regex}[1]{{\ttfamily#1}}

\newcommand{\impl}[2]{{\ttfamily#1/}{\ttfamily test/data/}{\ttfamily#2}}

% Speciální nastavení

% vycentrované floating objekty
% http://tex.stackexchange.com/questions/2651/should-i-use-center-or-centering-for-figures-and-tables
\makeatletter
\g@addto@macro\@floatboxreset\centering
\makeatother

\linenumbers
\modulolinenumbers[10]


%opening
\title{Aplikace fuzzy a pravděpodobnostních automatů, poznámky}
\author{Martin Jašek}
\date{12. září 2016 --- ??}

\begin{document}

\maketitle
\tableofcontents

\newpage

\section{Vybrané aplikace, poznámky}

%%%%%%%%%%%%%%%%%%%%%%%%%%%%%%%%%%%%%%%%%%%%%%%%%%%%%%%%%%%%%%%%%%%%%%%%%%%%%%%
\subsection{Fuzzy jazyky, gramatiky a regulární výrazy}


\subsubsection*{Fuzzy finite automata and fuzzy regular expressions with membership values in lattice-ordered monoids \cite{LiPed-FuzzFinAutFuzzRegExMembValLattOrdMon}}

Fuzzy regulární výrazy, to je taky tak trochu aplikace, ne? Bohužel tento článek taky není k dispozici ...


\subsubsection*{Minimization of lattice finite automata and its application to the decomposition of lattice languages \cite{LiPed-MinLattFinAutAppDecoLattLang}}

Tady se asi trošku zabředne trochu víc do matematiky, ale -- uvidíme. Uvidíme, co vyleze z dekompozice fuzzy jazyků.

\begin{definition}
 \TODO{lattice language} ($\sim$ fuzzy jazyk nad svazem)?
\end{definition}

\subsubsection*{Fuzzy languages with infinite range accepted by fuzzy automata: Pumping Lemma and determinization procedure \cite{GonGar-FuzzLangInfRanAccGuzzAutPumLemDetProc}}

Fuzzy pumping lemma. Podívat se na to.



\subsubsection*{Construction of fuzzy automata from fuzzy regular expressions \cite{StaCir-ConsFuzzAutFuzzRegExp}}

\subsubsection*{A probabilistic model of computing with words \cite{QiuWan-ProbModCompWords}}
Nový pohled na pravděpodobnostní automaty. Nepracují se symboly, ale celými slovy abecedy. Prý protože u slov je přirozenější uvažovat pravděpodobnosti výskytu. Slovo je v jejich případě rozložením pravděpodobnosti. Co to sakra ...?


\subsubsection*{Characterization of Fuzzy Regular Languages, \cite{RamGir-CharFuzzRegLang}}
Vztah fuzzy regulárních jazyků a fuzzy automatů. 

\subsubsection*{Fuzzy pushdown automata, \cite{BucPas-FuzzPushAut}}

Článek, který bohužel nejspíš není k dispozici, nicméně v abstraktu padá jedna důležitá myšlenka:

%It is shown that the B-fuzzy pushdown automata can accept context-sensitive languages by setting a threshold, while the (fuzzy) pushdown automata can accept only context-free languages.
B-fuzzy automaty přijímají kontextově závislé jazyky tak, že se jim nastaví práh. \uv{Obyčejné} fuzzy zásobníkové automaty přijímají jen bezkontextová jazyky.

\subsection{Pattern recognition (řetězců, základní techniky)}

\begin{definition}[Problém Pattern recognition]
\TODO{zjistit, co to vlastně PŘESNĚ doopravdy je}
\end{definition}

\begin{definition}[Syntactic Pattern recognition]
Mělo by to být něco ve smyslu, že patternem je klasický řetězec z terminálů (a né třeba gramatika).
\end{definition}


\TODO{rozbrat vzdálenosti (Hammingova, Levenshteinova)}


\subsubsection*{Conversion of Finite Automata to Fuzzy Automata for String Comparison \cite{RamGir-ConvFinAutFuzzAutStrComp}}

Jedná se o pravděpodobně nejjednodušší zavedení fuzzy automatu pro pattern recognition.

Mějme klasický bivalentní automat $M = (Q, \Sigma, q_0, \delta, F)$ a funkci $g_s: \Sigma \times \Sigma \rightarrow [0,1]$ (podobnost symbolů). Označme $\delta_a (q_i, q_j)$ jako $\delta(q_i, a, q_j)$.

Pak můžeme zkonstruovat fuzzy automat $M' = (Q, f, I, F')$ takový, že $I$ a $F'$ jsou počáteční a koncové stavy ($I(q_0) = 1$, $I(q_i) = 0$ pro $i > 0$, $F'(q_n) = 1$, $F(q_i) = 0$ pro $i < n$) a $f$ je fuzzy přechodová funkce:
$$
  f(q_i, a, q_j) = \bigvee_{x \in \Sigma} (g_s(a, x) \wedge \delta_x(q_i, q_j))
$$

Toť vše. Nevýhody jsou zřejmé -- $M'$ nám například nepokryje vložení nebo naopak odebrání symbolu.

\subsubsection*{Fuzzy automata with $\epsilon$-moves compute fuzzy measures between strings \cite{AstGonMenGar-FuzzAutEpsMovCmpFuzzMeasBtwStrs}}

Mějme fuzzy automat s $\epsilon$ přechody. Je dokázáno (viz např. článek), že rozpoznává stejný jazyk jako některý NFA (bez $\epsilon$ přechodů).

Autoři dále zavádějí nový pohled na rozdíly mezi dvěma řetězci. Používájí k tomu relaci $E = (\Sigma \cup \{ \epsilon \})^2$ porovnávající dvojici symbolů (resp. symbol nebo prázdný řetězec). 

\begin{definition}[Editační operace]
Mějme relaci $E = (\Sigma \cup \{ \epsilon \})^2$ (pospanou výše). Pak každou dvojici $z \in E$ nazvěme editační operace. Speciálně pak, $(x, y) \in E$ znamená nahrazení symbolu $x$ symbolem $y$, $(x, \epsilon) \in E$ znamená odebrání symbolu $x$ a naopak $(\epsilon, y) \in E$ pak vložení symbolu $y$ (pro všechny $x, y \neq \epsilon$). 

Máme-li editační operaci $(x, y) = z \in E$, pak označme $x = z^\downarrow$ a $y = z^\uparrow$.
\end{definition}

Podíváme-li se na $E$ jako na množinu (uspořádaných dvojic), můžeme ji tak považovat za abecedu. $E^+$ pak značí řetězce nad touto abecedou, tedy posloupnosti editačních operací dvou řetězců. Pomocí takovéto posloupnosti editačních operací můžeme popsat způsob, jak z jednoho řetězce vytvořit jiný. Takovéto posloupnosti editačních operací se říká vyrovnání řetězců.

\begin{definition}[Vyrovnání]
Máme-li řetězec $\omega$, který vznikl z řetězce $\alpha$ použitím posloupnosti $z \in E^+$ editačních sybolů, nazýváme vyrovnáním $\alpha$ k $\omega$.
\end{definition}

\begin{definition}[Množina vyrovnání]
Množinou $G(\alpha, \omega)$ vyrovnání rozumíme množinu všech vyrovnání $\alpha$ k $\omega$. \TODO{opsat přesnou definici (def. 7, str. 5)}
\end{definition}
Množina $G(\alpha, \omega)$ nám reprezenuje všechny možné způsoby, jak přetransformovat řetězec $\alpha$ na řetězec $\omega$. 

Nyní mějme fuzzy relaci $R$ nad $E$ (\TODO{reflexivita, symetrie a T-transitivita $R$}). Tato relace každé editační operaci $z \in E$ přiřazuje hodnotu \uv{váhy}. Na základě této fuzzy relace můžeme určit stupeň podobnosti dvojice řetězců $\alpha$ a $\omega$. Definujeme proto fuzzy zobrazení $S_{\Sigma, R, T}$ nad $\Sigma^*$ jako fuzzy míru mezi dvěma řetězci:
\begin{definition}[Fuzzy míra]
Fuzzy míra řetězci $\alpha$ a $\omega$ je fuzzy relace $S_{\Sigma, R, T}$ nad $\Sigma^*$ taková, že ... \TODO{opsat přesně} přes všechna $z \in G(\alpha, \omega)$ počítá maximum z $t$-normy $R(z_i)$ pro všechny editační operace $z_i \in z$.
\end{definition}

Nyní je třeba zkonstruovat automat (fuzzy automat s $\epsilon$-přechody). \TODO{opsat a okomentovat tu dlouhou definici}. Je dokázáno, že výpočet tohoto automatu koresponduje s fuzzy mírou $S_{\Sigma, R, T}$ definovanou výše.


\subsubsection*{Approximate String Matching Using Deformed Fuzzy Automata: A Learning Experience \cite{AstGariGonVillFar-ApprStrMatUsiDefFuzzAutLearExpr}, Deformed fuzzy automata for correcting imperfect strings of fuzzy symbols \cite{GarMenEchAstFar-DefFuzzAutCorImpStrFuzzSyms}}

\begin{definition}[Fuzzy symbol]
Fuzzy symbol je fuzzy množina všech symbolů abecedy. Její membership funkce je definována jako podobnost aktuálního symbolu s pozorovacným symbolem. Značí se typicky $\widetilde{y}$ pro pozorovaný symbol $y$.
\end{definition}

Používá následující algoritmus: Nejdříve se sestaví slovník, kde pro každou (výstupní) třídu se vytvoří (resp. vybere jeden z objektů v této třídě) její reprezentant. Poté se pro vstupní řetězec se počítá \uv{počet} transformací (přidání, odebrání nebo záměn symbolů) tak, abychom získali jednotlivé reprezentanty. Výslednou třídou vstupního řetězce se pak stane ta, jejiž reprezentant má ke vstupnímu řetězci \uv{nejblíž} (tj. vyžadoval nejméně transformací).

Otázkou však je, jak tyto transformace zpočítat. Základní technika, zvaná recognition rate \TODO{dohledat více informací o RR}, má však jisté nevýhody.  Např. věty uvažuje samostatně, bez kontextu okolních vět. Stejně tak slova. Proto se zavádí použití fuzzy symbolů.

Přechodem z \uv{běžných} symbolů na fuzzy symboly nám však přestává dostačovat klasický automat. Je proto nutné použít mocnější náststroj. Autoři proto používají fuzzy deformovaný automat.

Fuzzy deformovaný automat je zobecnění klasického automatu. Pro naše účely je vhodné vytvořit jej z automatu rozpoznávající řětězec vzoru (patternu). 

\begin{definition}[Automat rozpoznávající $\omega$, \cite{AstGariGonVillFar-ApprStrMatUsiDefFuzzAutLearExpr} str. 2]
  Mějme řetězec $\omega$ nad abecedou $\Sigma$. Automat rozpoznávající $\omega$ je pak konečný automat $M(\omega) = \langle Q, \Sigma, \delta, q_0, \{ q_n \} \rangle$ takový, že jeho množina stavů $Q$ se sestává z právě $n$ (kde $n$ je délka řetězce $\omega$) stavů $q_0, \dots, q_n$, dále $q_0$ je počáteční a $q_n$ koncový stav a $\delta$ je přechodová funkce $Q \times Q \times \Sigma \rightarrow \{0, 1\}$. Přechodová funkce je definována následovně:
  $$
  \delta(q_{k-1}, q_k, a_k) = 
  \begin{cases}
      1	&\text{pokud $a_k$ je $k$-tý symbol v řetězci $\omega$} \\
      0	&\text{jinak}
  \end{cases}
  $$
\end{definition}

Takto definovaný automat zřejmě bude rozpoznávat jen a jen řetězec $\omega$ a žádný jiný. Nyní z něj sestavíme fuzzy deformovaný automat.

\begin{definition}[Fuzzy deformovaný automat, \cite{AstGariGonVillFar-ApprStrMatUsiDefFuzzAutLearExpr} str. 2]
  Mějme automat $M(\omega)$ přijímající $\omega$. Pak definujeme fuzzy deformovaný automat $MD(\omega)$ jako $\langle Q, \Sigma, \mu, \sigma, \eta, \widetilde{\mu} \rangle$, kde:
  \begin{itemize}
   \item $Q$ je množina stavů přejatá z $M(\omega)$,
   \item $\Sigma$ je abeceda přejatá z $M(\omega)$,
   \item $\sigma: Q \rightarrow [0, 1]$ jsou počáteční stavy (tedy $\sigma(q_0) = 1$ a $\sigma(q_i) = 0$ pro všechna ostatní $q_0 \neq q_i \in Q$),  
   \item $\eta:   Q \rightarrow [0, 1]$ jsou koncové stavy (tedy $\sigma(q_n) = 1$ a $\sigma(q_i) = 0$ pro všechna ostatní $q_n \neq q_i \in Q$),
   \item $\mu:    Q \times Q \times (\Sigma \cup \{ \epsilon \}) \rightarrow [0, 1]$, fuzzy přechodová funkce: 
    $$
      \mu(q, p, a)
      \begin{cases}
	= 1  		&\text{pokud $\delta(q, p, a) = 1$, $q, p \in Q$ a $a \in \Sigma$} \\
	\in [0, 1]  	&\text{pokud $a = \epsilon$} \\
	\in [0, 1] 	&\text{pokud $a \in \Sigma$} \\
	\in [0, 1] 	&\text{pokud $p = q$} \\
	= 0 		&\text{jinak}
      \end{cases}
    $$
    \item $\widetilde{\mu}: \mathcal{F}(Q) \times \mathcal{F}(\Sigma) \rightarrow Q \times [0, 1]$ je funkce pro výpočet následujícího stavu definována následovně \TODO{zkontrolovat, jestli jsem to pochopil dobře}:
    $$
    \widetilde{\mu}(\widetilde{P}, \widetilde{y}) = \{ \langle p, \mu \rangle \mid
      \mu = \bigoplus_{q \in Q} ( \bigoplus_{x \in \Sigma} ( 
	\mu(q, p, x) \otimes \mu_{\widetilde{y}} (x) )
      \otimes \mu_{\widetilde{P}}(q) )
      \text{ $\forall p \in Q$} \}
    $$
  \end{itemize}
\end{definition}

\TODO{rozebrat stupňovitost, stupně pravdivosti jednotlivých elementů a taak}

V definici přechodové funkce $\mu$ můžeme pozorovat zavedení kýžené transformace. Pokud se symbol na vstupu shoduje se symbolem ve vzoru (na stejném místě), aplikuje se pravidlo přejaté z přechodové funkce $\delta$ původního automatu. Není-li ze vstupu přečten žádný symbol (tj. $a = \epsilon$), pak nám tím přechodová funkce dovoluje symbol vzoru z pozorovaného slova vynechat. Naopak, pokud přečteme symbol, ale nepřesuneme se do jiného stavu, pak sledované slovo oproti vzoru obsahuje symbol $a$ navíc. Pokud jsme přečetli symbol $a$, který však ale nesplňuje podmínku s pravidlem $\delta$, pak pzorované slovo obsahuje na dané pozici jiný symbol, než vzorové slovo. Pokud žádná z těchto situací nenastane, pak ji odmítneme. 


\subsubsection*{Intuitionistic Fuzzy Automaton for Approximate String Matching \cite{RavChoTri-IntFuzzAutApprStrMatch}}

Uvažuje se intuistionický přístup. \TODO{zjistit, co to ten intuistionický přístup vlastně je} Jde o to, že se kromě zobrazení podobnosti, uvažuje také zobrazení nepodobnosti. To je prý lepší, než klasické fuzzy (s jednou membership funkcí).

V článku poté definují fuzzy jazyk (triviální), fuzzy regulérní jazyk, intuistionický fuzzy jazyk (místo jedné membership funkce má dvě) a následně intuistonický fuzzy regulární jazyk. Na základě techto pojmů poté definují fuzzy intuistonický automat. Jdou na to přes automat rozpoznávající $\omega$, podobně jako v \cite{AstGariGonVillFar-ApprStrMatUsiDefFuzzAutLearExpr}. Dále však ale definují fuzzy automat, což zřejmě bude jen jiné označení pro fuzzy deformovaný automatu z onoho článku.

Následně je definován fuzzy intuistonický automat. Ten má každou z memership funkcí adekvátně zdvojenu (\TODO{přepsat definici z článku, str. 32)}. Stejně tak, celý výpočet neuvažuje jednu hodnotu stupně, ale dvojici hodnot. Stejně tak vypadá i (intuistionický fuzzy) jazyk.

\subsubsection*{Vlastní nápady}
\begin{itemize}
 \item Fuzzy pattern matching BZK jazyků (např. automatická korekce chyby při zápisu aritmetického výrazu)
 \item něco na zůsob fuzzy lexikálního analyzátoru (pro slovo rozpozná pattern a slovo adekvátně zatřídí) $\Rightarrow$ slití automatů (rozpoznávajících jednotlivé patterny) $\Rightarrow$ vznikne velký automat $\Rightarrow$ fuzzy minimalizace. Jaký by to mělo dopad na správnost klasifikace?
\end{itemize}


%%%%%%%%%%%%%%%%%%%%%%%%%%%%%%%%%%%%%%%%%%%%%%%%%%%%%%%%%%%%%%%%%%%%%%%%%%%%%%%
\subsection{Pattern recognition (ostatní)}




\subsubsection*{Syntactic Methods in Pattern Recognition \cite{Fu-SynMethPattRec}}

Knížka (celá) o pattern recognizition pomocí formálních jazyků. Bohužel jen klasických automatů případně stochastických.


\subsubsection*{Fractionally Fuzzy Grammars \cite{PalYau-FracFuzzGramAppPatRec} (popř. \cite{MorMal-FuzzAutAndLangs} 10.7)}

Fractionally fuzzy grammar je speciální fuzzy gramatika, která se dobře hodí pro patern recognition. Prý jsou \uv{powerfull} a součastně \uv{easily parsed}. \TODO{V \cite{MorMal-FuzzAutAndLangs}, 10.7 je několik odkazů na další zdroje, prozkoumat!}.

V článku \cite{PalYau-FracFuzzGramAppPatRec} je pojem Fractionally fuzzy grammar (nejspíš) zaveden.

\subsubsection*{(Fuzzy cellular learning automata for lesion detection in retina images \cite{NejAzaAdeMohMir-FuzzCellLeaAutLesDetRetIma})}

\subsubsection*{Fuzzy Automata System with Application to target recognition based on image processing \cite{WuaPanHan-FuzzAutSysAppTarRecBasImaProc}}

Pokročilejší technika pro pattern recognition. Využívají pro rozpoznávání otisků prstů.



\subsubsection*{Fuzzy tree automata and syntactic pattern recognisition\cite{Lee-FuzzTreeAutSyntPattRec}, \cite{MogZadAme-NewDirInFuzzTreeAut}}
To bude něco na způsob R-stromů. Z pattern se udělá strom, který je rozpoznáván fuzzy tree automatem.

\subsubsection*{Fuzzy Models and Algorithms for Pattern Recognition and Image Processing \cite{BezKelKriPal-FuzzModAlgPattRecImaProc}}
Megaknížka o tom, že se fuzzy modely používají pro pattern recognition. O automatech tam toho moc není. Ale na straně 545 popisují kýženou historii výzkumu aplikací fuzzy automatů v oblasti pattern recognition.

\subsubsection*{RS image processing automata \cite{Xia-RSImaProcAut}}
Nový druh (\uv{fuzzy}) automatu vytvořený pro image processing. Stavy jsou matice se vzory a přechody mezi nimi jsou vlastně maticové operace na ně postupně aplikované. Zajímavé...

\subsubsection*{Syntactic Pattern Recognition for Seismic Oil Exploration \cite{Hua-SyntPattRecSeisOilExp}}
Používají pattern recognisition pomocí automatů (m.j. pomocí tree automatů) k hledání ložisk ropy.


%%%%%%%%%%%%%%%%%%%%%%%%%%%%%%%%%%%%%%%%%%%%%%%%%%%%%%%%%%%%%%%%%%%%%%%%%%%%%%%
\subsection{Neuronové sítě a fuzzy neuronové sítě}

\begin{definition}[Neuronová síť]
Neuronová síť je ... \TODO{z nějakého pěkného zdroje doplnit}
\end{definition}

\begin{definition}[Knowledge based neuronová síť \cite{TowSha-KnowBas-ArtNeuNet}]
Knowledge based neuronová síť \TODO{jak se to řekne česky?} je neuronová síť, jejiž počáteční konfigurace není vygenerována náhodně, ale sestavena ze znalostní báze.
\end{definition}

\begin{definition}[Rekurentní neuronová síť]
 \TODO{zjistit, asi neuronovka, která umožňuje při průchodu impulzu návrat zpět}
\end{definition}


\subsubsection*{Fuzzy Finite-State Automata Can Be Deterministically Encoded Into Recurent Neural Networks, \cite{OmlThoGil-FuzzFinStaAutCanDetEncIntRecNeuNet}}

Bylo prokázáno (vyjmenována hromada zdrojů na straně 4), že konečný automat může být reprezentován rekurentní neuronovou sítí. 



\subsubsection*{Application of Fuzzy Automata Theory and Knowledge Based Neural Networks for Development of Basic Learning Model \cite{DarAhmSin-AppFuzzAutTheKnBsNeuNetDevBasLeaMod}}

Článek je poněkud nepřehledný, doporočuji nahradit \cite{OmlThoGil-FuzzFinStaAutCanDetEncIntRecNeuNet} nebo zkusit vyhledat ještě jiný. Nicméně:

Z fuzzy automatu byla zkonstruována neuronová síť zvaná Fuzzy  Automata  based  Neural  Network (FANN). Prý znalostní pravidla převádí na fuzzy automaty a z nich poté konstruuje FANN.

\begin{definition}
 \TODO{Model of Learning}
 \TODO{Knowledge rule}
 \TODO{Rule extraction}
\end{definition}


Autoři FANN demonstrují na Urban Trafic modeling.

\begin{definition}[Urban Trafic modeling]
 je problém modelace infrastruktury, predikce budoucího vývoje. Inteligentní řízení provozu a podob.
\end{definition}

\subsubsection*{Fuzzy Neural Networks \cite{LeeLee-FuzzNeuNet}}

To bude nejspíš podobné jako \cite{DarAhmSin-AppFuzzAutTheKnBsNeuNetDevBasLeaMod}. 


%%%%%%%%%%%%%%%%%%%%%%%%%%%%%%%%%%%%%%%%%%%%%%%%%%%%%%%%%%%%%%%%%%%%%%%%%%%%%%%
\subsection{Strojové učení (ostatní)}

\subsubsection*{A Formulation of Fuzzy Automata and Its Application as a Model of Learning Systems \cite{WeeFu-FormFuzzzAutItsAppModLearSyss}}

Originální paper bohužel není k dispozici. Ale dle anotace je to jeden z prvních článků vůbec. Podobá se stochastickému automatu.

\begin{definition}
 \TODO{Stochastický automat}
\end{definition}

Autoři předvádějí použití \uv{fuzzy automatu} jako model strojového učení (používají zastaralý\RLY výraz \uv{Model of learning system}). Dále uvádějí hypotézu použití jako automatická kontrola a pattern recognizition \TODO{jak se to řekne česky?}.

Na konci zmiňují, že použití fuzzy automatu v těchto situacích přináší především jednoduchost (návrhu i průběhu výpočtu).

\subsubsection*{Redukce přeučení pomocí fuzzy minimalizace automatu konečného jazyka}
Nebo tak něco. Zkrátka vezmeme dataset jako konečný jazyk. Zkonstruujeme mu automat, ten fuzzy-zminimalizujeme, takže získáme nějaký přibližný model těchto dat.

%%%%%%%%%%%%%%%%%%%%%%%%%%%%%%%%%%%%%%%%%%%%%%%%%%%%%%%%%%%%%%%%%%%%%%%%%%%%%%%
\subsection{Fuzzy regulátory a řídící systémy}

\subsubsection*{Fuzzy Reasoning and Fuzzy Automata in User Adaptive Systems \cite{Kov-ReasFuzzAutInUsrAdpSyss}}

Řídící systémy, adaptivní systémy? Každopádně, bude to něco na principu fuzzy regulátorů. Nejspíš to bude fungovat tak, že vstupní řetězec bude posloupnost real--time událostí vně systému. A on na ně bude nějak reagovat. A něco dělat.

\begin{definition}
 \TODO{Fuzzy reazoning}
 \TODO{Fuzzy state transition rulebase}
\end{definition}

V příkladu uvádí nástroj pro výběr nejlepší židle. Uživatelé jsou požádání o označení jak moc na ně působí zvolené židle (co do modernosti, pohodlnnosti, ceny, ...) a systém vygeneruje uspořádání židlí od \uv{nejlepší} po \uv{nejhorší}.

Víc toho nevím, protože text působí hodně odborně a vyžaduje tak znalost základních pojmů.


\subsubsection*{A model for Finite-state probabilistic systems \cite{BruFu-ModelFinStateProbSyss}}

Na tento článek se odkazuje \cite{MorMal-FuzzAutAndLangs} v kapitole 10.1. Ale kromě anotace jsem se o něm nedozvěděl nic víc.


%%%%%%%%%%%%%%%%%%%%%%%%%%%%%%%%%%%%%%%%%%%%%%%%%%%%%%%%%%%%%%%%%%%%%%%%%%%%%%%


%%%%%%%%%%%%%%%%%%%%%%%%%%%%%%%%%%%%%%%%%%%%%%%%%%%%%%%%%%%%%%%%%%%%%%%%%%%%%%%
\subsection{Bioinformatika a medicína}




\subsubsection*{An application of intuitionistic fuzzy sets in medical diagnosis \cite{SupBisAkh-AppIntFuzzSetMedDiag}}

Zní to příšerně, ale nejspíš to bude vycházet z \cite{GupSar-FuzzAutDecProc}.


%%%%%%%%%%%%%%%%%%%%%%%%%%%%%%%%%%%%%%%%%%%%%%%%%%%%%%%%%%%%%%%%%%%%%%%%%%%%%%%
\subsection{Ostatní aplikace a zdroje}

\subsubsection*{Zmíňka o sadě aplikací v \cite{OmlThoGil-FuzzFinStaAutCanDetEncIntRecNeuNet}}
V \cite{OmlThoGil-FuzzFinStaAutCanDetEncIntRecNeuNet} na straně 4 mluví o tom, že fuzzy gramatiky mají široké použití (od analýzy rentgenových snímků, přes návrh elektronických obvodů až po návrh GUI). A ještě spoustu dalších. Včetně odkazů na zdroje!

\subsubsection*{An Adaptive Vehicle Path Planning System \cite{HuaHuWanCheTas-AdaVehPathPlanSys}}
Super!

\subsubsection*{Fuzzy automata and decision processes\cite{GupSar-FuzzAutDecProc}}

To zní lákavě, ale bohužel online není prakticky ani anotace.


\subsubsection*{An introductory survey of fuzzy control ??}

Podívat se na tohle, jestli se tam vůbec mluví o použití (fuzzy) automatů:

http://www.sciencedirect.com/science/article/pii/002002558590026X



\subsubsection*{Fuzzy evolutionary cellular automata \cite{Rich-FuzEvoCellAut}}
Jedná se vůbec o fuzzy automat? Resp. Mluví se tam o fuzzy teorie a celulárním automatu. Tak nevím.

%%%%%%%%%%%%%%%%%%%%%%%%%%%%%%%%%%%%%%%%%%%%%%%%%%%%%%%%%%%%%%%%%%%%%%%%%%%%%%%
%%%%%%%%%%%%%%%%%%%%%%%%%%%%%%%%%%%%%%%%%%%%%%%%%%%%%%%%%%%%%%%%%%%%%%%%%%%%%%%
\section{Jiná literatura}


%%%%%%%%%%%%%%%%%%%%%%%%%%%%%%%%%%%%%%%%%%%%%%%%%%%%%%%%%%%%%%%%%%%%%%%%%%%%%%%
\subsection{Něco k teorii (různé definice automatů, determinizmus, minimalizace, ekvivalence, ...)}

\subsubsection*{Determinism and fuzzy automata \cite{Bel-DetFuzzAut}}
Článek určitě zajímavý do teoretické části, ale dost možná se bude zmiňovat o (alespoň!) některých aplikacích.

\subsubsection*{A Note on Fuzzy Tree Automata \cite{ChaJos-NoteFuzzTreeAut}}
Pěkný skromný ale obsáhlý článek o fuzzy tree automata.

\subsubsection*{A formal model of computing with words \cite{Yin-FormModCompWords}}
Uveden nový fuzzy automat, který místo řetězců nad abecedou přijímá řetězce nad fuzzy-podmnožinami abecedy. Prý je to hned v několika oblastech lepší.

\subsubsection*{On fuzzy multiset automata}
http://search.proquest.com/computerscience/docview/1784418314/55737592510249A1PQ/6?accountid=16730

\subsubsection*{Bisimulations for fuzzy automata}
http://search.proquest.com/computerscience/docview/963845527/55737592510249A1PQ/5?accountid=16730


%%%%%%%%%%%%%%%%%%%%%%%%%%%%%%%%%%%%%%%%%%%%%%%%%%%%%%%%%%%%%%%%%%%%%%%%%%%%%%%
\subsection{Koncepce (fuzzy a/vs. pravděpodobnostní přístup)}

\subsubsection*{Do exact shapes of fuzzy sets matter? \cite{Bel-DoExaShaFuzzSetMatt}}
Úvaha nad tím, jestli je vlastně OK používat fuzzy teorii. Protože fuzzy teorie sice zavádí trošku neurčitosti, ale pořád je tato neurčitost popsána matematicky.



%%%%%%%%%%%%%%%%%%%%%%%%%%%%%%%%%%%%%%%%%%%%%%%%%%%%%%%%%%%%%%%%%%%%%%%%%%%%%%%
\subsection{Další čtení a potencionální zdroje}
\begin{itemize}
 \item Morderson, Malik: Fuzzy Automata and Languages: Theory and Applications \cite{MorMal-FuzzAutAndLangs}
 \item Introduction to Probabilistic Automata \cite{Paz-IntroProbAut}, samostatná kapitola se seznamem vybraných aplikací pravděpodobnostních automatů
 \item New directions in fuzzy automata \cite{DooKre-NewDirFuzzAut}: Nový pohled na automaty, v anotaci mluví o tom, jak dlouho se automaty již studují, takže by se tam mohl odkazovat na nějaké aplikace. Aplication driven methodology?
 \item Inverse fuzzy automata and inverse fuzzy languages, http://www.afmi.or.kr/articles\_in\_%20press/2013-03/AFMI-H-121231/AFMI-H-121231.pdf
 \item Entropies of probabilistic grammars, http://www.sciencedirect.com/science/article/pii/S0019995874907992
 \item Pattern Recognition and Machine Learning, http://bit.ly/2cD2FO3
 \item Applied Automata Theory (hlavně pravděpodobnostní) http://bit.ly/2cCHJ8i
 \item PRISM: A Tool for Automatic Verification of Probabilistic Systems, http://eprints.gla.ac.uk/43841/1/43841.pdf,
 \item verifikace a checking of nondeterminism pravděpodobnostních systémů tak nějak jakože všeobecně
 \item A Bibliography on Fuzzy Automata, Grammars and Lanuages, anotace: http://doc.utwente.nl/64296/1/MI-95-46.pdf
 \item Approximate String Matching by Fuzzy Automata, anotace: http://link.springer.com/chapter/10.1007/978-3-642-00563-3\_29
 \item Approximate String Matching Using Deformed Fuzzy Automata: A Learning Experience, anotace: http://link.springer.com/article/10.1023/B%3AFODM.0000022042.64558.1d
 \item Klasická automata theory: J.E. Hopcroft, J.D. Ullman, Introduction to Automata Theory, Languages, and Computation, Addison-Wesley, Reading,
MA, 1979
 \item Latent Dependency Forest Models, https://www.researchgate.net/publication/307931171\_Latent\_Dependency\_Forest\_Models
 \item A Bibliography on Fuzzy Automata, Grammars and Languages (mega seznam článků): http://doc.utwente.nl/64296/1/MI-95-46.pdf
 
 
 \end{itemize}

Knížky, které se zmiňují o fuzzy teorii, automatech a/nebo aplikacích:
\begin{itemize}
 \item Cornelius T Leondes: Fuzzy theory systems. 1999.  
 \item Rudolf Seising: The Fuzzification of systems : the genesis of fuzzy set theory and its initial applications - developments up to the 1970s. 2007
 \item C B Setzer; N A Warsi: Fuzzy automata and pattern matching : final report. 1986
 \item Chichester, West Sussex: Foundations of fuzzy control : a practical approach. 2013
 \item John H Lilly: Fuzzy control and identification. 2010
 \item Hugh M Cartwright; Nawwaf Kharma: Using artificial intelligence in chemistry and biology : a practical guide, 2008
 \item Stan Openshaw; Robert J Abrahart: GeoComputation, 2000
 \item Friedrich Recknagel: Ecological informatics : scope, techniques, and applications, 2006
 \item Roberto Cipolla; Sebastiano Battiato; Giovanni Maria Farinella: Computer vision : detection, recognition and reconstruction, 2010

 \item National Research Council (U.S.). Transportation Research Board: Traffic flow theory and characteristics. Volume 2, 2015
 \item John N Mordeson; D S Malik; Shiquan Cheng: Fuzzy mathematics in medicine
 \item Ali Saghafinia: Applications of various fuzzy sliding mode controllers in induction motor drives, 2016
 \item David M Forrester: Fuzzy cellular automata in conjunctive normal form., 2011
 \item Jelena Ignjatović, Miroslav Ćirić, Stojan Bogdanović: Determinization of fuzzy automata with membership values in complete residuated lattices, 2008
 \item Miroslav Ćirić, Jelena Ignjatović, Nada Damljanović, Milan Bašić: Bisimulations for fuzzy automata (2012)
 \item Jianhua Jin, Qingguo Li: Fuzzy grammar theory based on lattices (2012)
\end{itemize}

\subsection{Třeba by mohlo být užitečné ...}
\begin{itemize}
 \item Social Cognitive Learning Theory and other Theories and Models, https://www.learning-theories.com/
 \item Zadeh: Fuzzy algorithms, http://www.sciencedirect.com/science/article/pii/S0019995868902118
\end{itemize}

\newpage
\bibliography{resources}
\bibliographystyle{plain}


\end{document}
