\documentclass[a4paper,10pt]{article}
\usepackage[utf8x]{inputenc}
\usepackage[czech]{babel}
\usepackage{amsthm}
\usepackage{amsmath}
\usepackage{amsfonts}
\usepackage{mathtools}

%% číslovaný cases
\usepackage[subnum]{cases}

%% colspan a rowspan v tabulkách
\usepackage{multirow}

%%  http://tex.stackexchange.com/a/361157
\usepackage{showlabels}

%% subfigures, https://en.wikibooks.org/wiki/LaTeX/Floats,_Figures_and_Captions#Subfloats
\usepackage{subcaption}

%% číslování řádků, https://texblog.org/2012/02/08/adding-line-numbers-to-documents/
\usepackage{lineno}

%Definice, věta, důkaz!
\newtheorem{definition}{Definice}
\newtheorem{theorem}{Věta}
\newtheorem*{notation*}{Značení}
\newtheorem{note}{Poznámka}
\newtheorem{example}{Příklad}


\numberwithin{definition}{section}
\numberwithin{theorem}{section}
%\numberwithin{notation}{section}
\numberwithin{note}{section}
\numberwithin{example}{section}


%Ostatní makra
\newcommand{\TODO}[1]{ \textit{\small(zde bude doplněno: #1)} }	
\newcommand{\RLY}{?? }

\newcommand{\lattL}{\mathbb{L}}
\newcommand{\fsubsets}[1]{\mathcal{F}(#1)}

\newcommand{\term}[1]{\textit{#1}}
\newcommand{\str}[1]{{\ttfamily#1}}
\newcommand{\regex}[1]{{\ttfamily#1}}

\newcommand{\impl}[2]{{\ttfamily#1/}{\ttfamily test/data/}{\ttfamily#2}}

% Speciální nastavení

% vycentrované floating objekty
% http://tex.stackexchange.com/questions/2651/should-i-use-center-or-centering-for-figures-and-tables
\makeatletter
\g@addto@macro\@floatboxreset\centering
\makeatother

\linenumbers
\modulolinenumbers[10]


%opening
\title{Fuzzy automaty}
\author{Martin Jašek}
\date{12. října 2014}

\begin{document}

\maketitle
\tableofcontents

\newpage

\section{Fuzzy automaty}
\TODO

Zkusit použít zdroje
\cite{fuzz-aut-and-langs}
\cite{fuzz-db-model-xml}
\cite{fuzz-grammar}
\cite{fuzz-rel-syss}
\cite{incompl-info-rough-set-anal},
fuzzy čísla
.



\section{Aplikace fuzzy automatů}
\TODO

Zkusit použít zdroje
\cite{fuzz-log-in-eng}
\cite{fuzz-met-rozh}
\cite{fuzz-tech-in-img-proc}
\cite{incompl-info-rough-set-anal},
Fuzzy surfaces in GIS and geographical analysis,
Hodnocení variant metodou fuzzy váženého průměru,
fuzzy Petri nets,
hodnocení efektivnosti,
kvalitativní politologický výzkum,
psychologie,
fuzzy analýza,
NEFRIT (univerzální nástroj pro multikriteriální hodnocení a rozhodování na bázi fuzzy logiky),
Odhad informace z dat vágní povahy,
jazykově orientované fuzzy modelování
.

\newpage
\bibliography{resources0}
\bibliographystyle{plain}


\end{document}
