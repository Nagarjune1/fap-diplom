\documentclass[a4paper,10pt]{article}
\usepackage[utf8x]{inputenc}
\usepackage[czech]{babel}

\usepackage{amsthm}
\usepackage{amsmath}
\usepackage{amsfonts}
\usepackage{mathtools}

%% číslovaný cases
\usepackage[subnum]{cases}

%% colspan a rowspan v tabulkách
\usepackage{multirow}


%Definice, věta, důkaz!
\newtheorem{definition}{Definice}
\newtheorem{note}{Poznámka}
\newtheorem{example}{Příklad}


\numberwithin{definition}{section}
\numberwithin{note}{section}
\numberwithin{example}{section}


%Ostatní makra
\newcommand{\TODO}[1]{ \textit{(zde bude doplněno: #1)} }	
\newcommand{\RLY}{?? }

\newcommand{\lattL}{\mathbb{L}}
\newcommand{\fsubsets}[1]{\mathcal{F}(#1)}

\newcommand{\term}[1]{\textit{#1}}
\newcommand{\str}[1]{{\ttfamily#1}}


\title{Nové poznámky k diplomce}
\author{Martin Jašek}
\date{5. červenec 2017 --- ??}

\begin{document}

\maketitle
\tableofcontents
\newpage

\section{Aktuální seznam nalezených technik}

\begin{itemize}
 \item deformace a modifikace klasického automatu pro rozpoznávání řetězců
 \item převod Fuzzy automatu na neuronovou síť
 \item fuzzy buněčné automaty
 \item Fuzzy Tree automaty
 \item učení automatu (PedGac-LeaFuzzAut)
\end{itemize}

\section{Aktuální seznam reálných aplikací}
\begin{itemize}
 \item Podobnostní porovnávání řetězců, určení nejpodobnějšího (deformace a modifikace klasického automatu pro rozpoznávání řetězců)
 \item hledání hran (pomocí buněčných automatů)
 \item rozpoznávání domečku (pomocí Fuzzy Tree automatu)
 \item Petriho sítě (???)
 \item urban trafic problem (???)
 \item city growt problem (???)
\end{itemize}

\section{Další možné aplikace}
\begin{itemize}
 \item Reprezentace fraktálovitých obrazů dle [Kru]
 \item Lindenmayer system
 \item Fuzzy DNA Motifs ?
 \item zobecnit jakékoliv aplikace klasických automatů?
 \item uměle se snažit vpasovat automaty na klasické fuzzy problémy?
 \item zaměřit se i na buněčné?
 \item co třeba porovnávání množin (nebo množin množin)? $\Rightarrow$ Fuzzy Tree automaty? (má to logicky smysl? množiny jsou exaktní, proč přibližné porovnávání?)
 \item přechodové systémy, distribuované systémy, něco s nima?
 \item teorie her? nějaký model pro AI hráče?
 \item výpočty, aproximace, rozklad? (čísel z $[0,1]$)
 \item problém inference gramatiky (z jazyka)
 \item výpočty s celými slovy
\end{itemize}

\section{Vlastní návrhy}
\begin{itemize}
 \item fuzzy \uv{binární} klasifikace (detekce síl hesla), de facto se jedná o problém inference gramatiky
 
\end{itemize}


\end{document}
