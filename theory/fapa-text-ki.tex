%%%  Ukázkový text a dokumentace stylu pro text závěrečné (bakalářské a
%%%  diplomové) práce na KI PřF UP v Olomouci
%%%  Copyright (C) 2012 Martin Rotter, <rotter.martinos@gmail.com>
%%%  Copyright (C) 2014 Jan Outrata, <jan.outrata@upol.cz>


%%  Pro získání PDF souboru dokumentu je třeba tento zdrojový text v
%%  LaTeXu přeložit (dvakrát) programem pdfLaTeX.

%%  V případě použití programu BibLaTeX pro tvorbu seznamu literatury
%%  je poté ještě třeba spustit program Biber s parametrem jméno
%%  souboru zdrojového textu bez přípony a následně opět (dvakrát)
%%  přeložit zdrojový text programem pdfLaTeX.

%%  Postup získání Postscriptového souboru je popsán v dokumentaci.


%%  Třída dokumentu implementující styl pro závěrečnou práci. Vybrané
%%  nepovinné parametry (ostatní v dokumentaci):

%%  'master' pro sazbu diplomové práce, jinak se sází bakalářská práce

%%  'field=kód' pro Váš studijní obor, kódy pro diplomovou práci 'uvt'
%%  pro Učitelství výpočetní techniky pro střední školy a 'binf' pro
%%  Bioinformatiku, jinak je výchozí Informatika, a pro bakalářskou
%%  práci 'ainfk' pro Aplikovanou informatiku v kombinované formě,
%%  'inf' pro Informatiku, 'infv' pro Informatiku pro vzdělávání a
%%  'binf' pro Bioinfomatiku, jinak je výchozí Aplikovaná informatika
%%  v prezenční formě

%%  'printversion' pro sazbu verze pro tisk (nebarevné logo a odkazy,
%%  odkazy s uvedením adresy za odkazem, ne odkazy do rejstříku),
%%  jinak verze pro prohlížeč

%%  'biblatex' pro zapnutí podpory pro sazbu bibliografie pomocí
%%  BibLaTeXu, jinak je výchozí sazba v prostředí thebibliography

%%  'language=jazyk' pro jazyk práce, jazyky english pro anglický,
%%  slovak pro slovenský, jinak je výchozí czech pro český

%%  'font=sans' pro bezpatkový font (Iwona Light), jinak výchozí
%%  patkový (Latin Modern)

\documentclass[
  master,
  field=inf,
%  printversion,
  biblatex,
  language=czech,
%  font=sans,
%  glossaries,
  index
]{kidiplom}

%% Informace pro úvodní strany. V jazyku práce (pokud není v komentáři
%% uvedeno česky) a anglicky. Uveďte všechny, u kterých není v
%% komentáři uvedeno, že jsou volitelné. Při neuvedení se použijí
%% výchozí texty. Text pro jiný než nastavený jazyk práce (nepovinným
%% parametrem language makra \documentclass, výchozí český) se zadává
%% použitím makra s uvedením jazyka jako nepovinného parametru.

%% Název práce, česky a anglicky. Měl by se vysázet na jeden řádek.
\title{Aplikace fuzzy a pravděpodobnostních automatů}
\title[english]{Applications of fuzzy and probabilistic automata}

%% Volitelný podnázev práce, česky a anglicky. Měl by se vysázet na
%% jeden řádek. Výchozí je prázdný.
%\subtitle{}
%\subtitle[english]{}

%% Jméno autora práce. Makro nemá nepovinný parametr pro uvedení
%% jazyka.
\author{Martin Jašek}

%% Jméno vedoucího práce (včetně titulů). Makro nemá nepovinný
%% parametr pro uvedení jazyka.
\supervisor{doc. Michal Krupka, PhD.}

%% Volitelný rok odevzdání práce. Výchozí je aktuální (kalendářní)
%% rok. Makro nemá nepovinný parametr pro uvedení jazyka.
%\yearofsubmit{\the\year}

%% Anotace práce, včetně anglické (obvykle překlad z jazyka
%% práce). Jeden odstavec!
\annotation{Tato práce pojednává o aplikacích fuzzy a pravděpodobnostních automatů.}

\annotation[english]{This thesis is about applications of fuzzy and probabilistic automata.}

%% Klíčová slova práce, včetně anglických. Oddělená (obvykle) středníkem.
\keywords{automaty; fuzzy množiny; pravděpodobnostní počet; aplikace}
\keywords[english]{automata; fuzzy sets; probability; applications}

%% Volitelná specifikace příloh textu práce, i anglicky. Výchozí je '1
%% CD/DVD'.
%\supplements{jedno kulaté placaté CD/DVD s malou kulatou dírou uprostřed}
%\supplements[english]{one round flat CD/DVD with a small round hole in the middle}

%% Volitelné poděkování. Stručné! Výchozí je prázdné. Makro nemá
%% nepovinný parametr pro uvedení jazyka.
\thanks{Poděkování patří vedoucímu práce, doc. Krupkovi.}

%% Cesta k souboru s bibliografií pro její sazbu pomocí BibLaTeXu
%% (zvolenou nepovinným parametrem biblatex makra
%% \documentclass). Použijte pouze při této sazbě, ne při (výchozí)
%% sazbě v prostředí thebibliography.
\bibliographystyle{plain}
\bibliography{resources.bib}

%% Další dodatečné styly (balíky) potřebné pro sazbu vlastního textu
%% práce.
%%%%%%%%%%%%%%%%%%%%%%%%%%%%%%%%%%%%%%%%%%%%%%%%%%%%%%%%%%%%%%%%%%%%%
%% různé matematické package
\usepackage{amsthm}
\usepackage{amsmath}
\usepackage{amsfonts}
\usepackage{amssymb}
\usepackage{mathrsfs}
\usepackage{mathtools}

%% číslovaný cases
\usepackage[subnum]{cases}

%% colspan a rowspan v tabulkách
\usepackage{multirow}

%%  http://tex.stackexchange.com/a/361157
\usepackage{showlabels}

%% subfigures, https://en.wikibooks.org/wiki/LaTeX/Floats,_Figures_and_Captions#Subfloats
\usepackage{subcaption}

%% číslování řádků, https://texblog.org/2012/02/08/adding-line-numbers-to-documents/
\usepackage{lineno}

%% chorvatská jména autorů, https://tex.stackexchange.com/questions/14872/croatian-serbian-letters-in-cv-problem
\usepackage[T1]{fontenc}

%% https://tex.stackexchange.com/questions/4465/put-a-slash-through-a-word?rq=1
%% \usepackage{cancel}

%% \verb , https://tex.stackexchange.com/questions/2790/when-should-one-use-verb-and-when-texttt
%\usepackage{verbatim}

%Definice, věta, důkaz!
%\newtheorem{definition}{Definice}
%\newtheorem{theorem}{Věta}
\newtheorem*{notation*}{Značení} %\TODO{nahrazeno 'notation'}
\newtheorem{notation}{Značení}
\newtheorem{note}{Poznámka}
%\newtheorem{example}{Příklad}


\numberwithin{definition}{section}
\numberwithin{theorem}{section}
%\numberwithin{notation}{section}
\numberwithin{note}{section}
\numberwithin{example}{section}


%Ostatní makra
\newcommand{\TODO}[1]{$\ast$} 	%\textit{\small(zde bude doplněno: #1)}} 
\newcommand{\RLY}{?? }

\newcommand{\lattL}{\mathbb{L}}
\newcommand{\fsubsets}[1]{\mathcal{F}(#1)}

\newcommand{\term}[1]{\textit{#1}}
\newcommand{\str}[1]{{\ttfamily#1}}
\newcommand{\regex}[1]{{\ttfamily#1}}
\newcommand{\ifthen}{\textsc{If--Then} }

\newcommand{\impl}[2]{{\ttfamily#1/}{\ttfamily test/data/}{\ttfamily#2}}

%%%%%%%%%%%%%%%%%%%%%%%%%%%%%%%%%%%%%%%%%%%%%%%%%%%%%%%%%%%%%%%%%%%%%

\begin{document}
%% Sazba úvodních stran -- titulní, s bibliografickými údaji, s
%% anotací a klíčovými slovy, s poděkováním a prohlášením, s obsahem a
%% se seznamy obrázků, tabulek, vět a zdrojových kódů (pokud jejich
%% sazba není vypnutá).
\maketitle

%% -------------------------------------------------------------------

\include{fapa-text}
%% -------------------------------------------------------------------

%% Vlastní text závěrečné práce. Pro povinné závěry, před přílohami,
%% použijte prostředí kiconclusions. Povinná je i příloha s obsahem
%% přiloženého CD/DVD.

%% Závěry práce. V jazyce práce a anglicky. Text pro jiný než
%% nastavený jazyk práce (nepovinným parametrem language makra
%% \documentclass, výchozí český) se zadává použitím makra s uvedením
%% jazyka jako nepovinného parametru.
\begin{kiconclusions}
Závěr práce v \uv{českém} jazyce.
\end{kiconclusions}

\begin{kiconclusions}[english]
Thesis conclusions in \uv{English}.
\end{kiconclusions}

%% Přílohy obsahu textu práce, za makrem \appendix.
\appendix

%% Obsah přiloženého CD/DVD. Poslední příloha. Upravte podle vlastní
%% práce!
\section{Obsah přiloženého CD/DVD} \label{sec:ObsahCD}



\begin{description}

\item[\texttt{bin/}] \hfill \\
  Instalátor \textsc{Instalator} programu, popř.~program
  \textsc{Program}, spustitelné přímo z~CD/DVD. / Kompletní adresářová
  struktura webové aplikace \textsc{Webovka} (v~ZIP archivu) pro
  zkopírování na webový server. Adresář obsahuje i~všechny runtime
  knihovny a~další soubory potřebné pro bezproblémový běh instalátoru
  a~programu z~CD/DVD / pro bezproblémový provoz webové aplikace na
  webovém serveru.

\item[\texttt{doc/}] \hfill \\
  Text práce ve formátu PDF, vytvořený s~použitím závazného stylu KI
  PřF UP v~Olomouci pro závěrečné práce, včetně všech příloh,
  a~všechny soubory potřebné pro bezproblémové vygenerování PDF
  dokumentu textu (v~ZIP archivu), tj.~zdrojový text textu, vložené
  obrázky, apod.

\item[\texttt{src/}] \hfill \\
  Kompletní zdrojové texty programu \textsc{Program} / webové aplikace
  \textsc{Webovka} se všemi potřebnými (příp.~převzatými) zdrojovými
  texty, knihovnami a~dalšími soubory potřebnými pro bezproblémové
  vytvoření spustitelných verzí programu / adresářové struktury pro
  zkopírování na webový server.

\item[\texttt{readme.txt}] \hfill \\
  Instrukce pro instalaci a~spuštění programu \textsc{Program}, včetně
  všech požadavků pro jeho bezproblémový provoz. / Instrukce pro
  nasazení webové aplikace \textsc{Webovka} na webový server, včetně
  všech požadavků pro její bezproblémový provoz, a~webová adresa, na
  které je aplikace nasazena pro účel testování při tvorbě posudků
  práce a~pro účel obhajoby práce.

\end{description}

%% -------------------------------------------------------------------

%% Sazba volitelného seznamu zkratek, za přílohami.
%\printglossary

%% Sazba povinné bibliografie, za přílohami (případně i za seznamem
%% zkratek). Při použití BibLaTeXu použijte makro
%% \printbibliography. jinak prostředí thebibliography. Ne obojí!

%% Sazba i v textu necitovaných zdrojů, při použití
%% BibLaTeXu. Volitelné.
\nocite{*}
%% Vlastní sazba bibliografie při použití BibLaTeXu.
\printbibliography

%% Bibliografie, včetně sazby, při nepoužití BibLaTeXu.
% \begin{thebibliography}{9}
%\bibitem{kniha2} \uppercase{Hawke}, Paul. NanoHttpd: Light-weight HTTP server designed for embedding in other applications. GitHub [online]. 2014-05-12. [cit. 2014-12-06]. Dostupné z: \url{https://github.com/NanoHttpd/nanohttpd}
%
%\bibitem{jeske13} \uppercase{Jeske}, David; \uppercase{Novák}, Josef. Simple HTTP Server in \csharp: Threaded synchronous HTTP Server abstract class, to respond to HTTP requests. CodeProject: For those who code [online]. 2014-05-24. [cit. 2014-12-06]. Dostupné z: \url{http://www.codeproject.com/Articles/137979/Simple-HTTP-Server-in-C}
%
%\bibitem{uzis2012} \uppercase{ÚSTAV ZDRAVOTNICKÝCH INFORMACÍ A STATISTIKY ČR}. Lékaři, zubní lékaři a farmaceuti 2012 [online]. Praha 2, Palackého náměstí 4: Ústav zdravotnických informací a statistiky ČR, 2012 [cit. 2014-12-06]. ISBN 978-80-7472-089-5. Dostupné z: \url{http://www.uzis.cz/publikace/lekari-zubni-lekari-farmaceuti-2012}
% \end{thebibliography}

%% Sazba volitelného rejstříku, za bibliografií.
%\printindex

\end{document}
